\documentclass{article}

\newcommand{\ie}{\textit{i.e.}}
\newcommand{\eg}{\textit{e.g.}}
\newcommand{\etal}{\textit{et al.}}
\newcommand{\etc}{\textit{etc.}}
\newcommand{\adhoc}{\textit{ad hoc}}

% Packages and abbreviations used by Konrad
\usepackage{latexsym}
\usepackage{amsmath}
\usepackage{amssymb}
\usepackage{amsbsy}
\usepackage{amsthm}

\newcommand{\konst}[1]{\ensuremath{\mathsf{#1}}}
\newcommand{\imp}{\Rightarrow}
\newcommand{\set}[1]{\ensuremath{\{ {#1} \}}}
\newcommand{\kstar}[1]{\ensuremath{{#1}^{*}}}
\newcommand{\Lang}[1]{\ensuremath{{\mathcal L}({#1})}}
\newcommand{\Angled}[1]{\ensuremath{\langle {#1} \rangle}}

\theoremstyle{definition}
\newtheorem*{example}{Example}
\newtheorem*{remark}{Remark}


\usepackage{graphicx}
\usepackage{tikz}
\usetikzlibrary{positioning}
\usetikzlibrary{arrows}
\usetikzlibrary{shapes.multipart}

\begin{document}


\title{Ideas for SafeDocs-CASE Interaction}
\author{Konrad Slind \\ Trusted Systems Group \\ Collins Aerospace}
\date{\today}

\begin{abstract}
The SafeDocs and CASE projects have shared interests in expressive
message description languages, secure parsing, and verified
implementations. We discuss some research ideas in this area,
confining ourselves to a technical discussion.
\end{abstract}

\maketitle

\section*{Introduction}

The present author gave a talk to the SafeDocs PI meeting (November
10, 2020), titled \emph{Specifying Message Formats with Contiguity
  Types}. The talk provided an overview of the CASE project with a
focus on work formalizing and proving correct a message format
language and its parsing algorithm. The ensuing discussion revealed an
overlap in research concerns that should be explored.

\emph{Contiguity types} provide a small but expressive DSL for
specifying message formats, especially those with
\emph{self-describing} aspects such as length fields and
`unions'. Filters and parsers can be automatically generated from
contiguity type specifications, backed up by a formal correctness
proof. The syntax of contiguity types is given in Figure \ref{ctypes}.
In CASE, contiguity types have been used to implement the
``insert-filter" and ``insert-monitor" architectural transformations for
the set of UxAS message types. These messages can be very large and
complex; we were pleased to see that contiguity type specifications
capture UxAS message structure and constraints transparently.

We feel that the following aspects of contiguity types and message
parsing seem worth pursuing in wider contexts:

\begin{description}

\item [First class `look-behind'] A common aspect of message parsing
  is that \emph{context}, \ie, already seen information in the
  message, is used to control further parsing steps. In our work, as a
  contiguity type parser processes a message, each field is added to
  the context (a finite map from C-style \emph{lvars} to message
  elements). This context is used in the computation of length fields,
  and in calculating which element of a union to
  choose. \footnote{Attribute grammars seem to also offer a uniform
    naming scheme for accessing context. The relationship needs more
    study and we have been talking to the SRI SafeDocs performers
    about this.}

\item [In-message assertions] Message constraints can be expressed
  within a contiguity type. This allows data constraints to be
  expressed alongside the data in the message specification. In an
  implementation, this can be leveraged to provide fast failure and
  copy-free filtering.

\item [Context-sensitive sum] There is no unadorned `sum' operation
  representing the union of two formal languages in the syntax of
  contiguity types. Instead, there is a `guarded union` type
  \[
    \konst{Union}\; (\mathit{bexp}_1 : \tau_1) \ldots (\mathit{bexp}_n : \tau_n)
  \]
  where evaluation of the $\mathit{bexp}_i$, in the accumulated
  context, is used to determine which $\tau_i$ to continue parsing
  with. There are similar notions in other parser frameworks, \eg, PEG
  parsers and ANTLR, but those designs focus on \emph{lookahead}
  information, handling look-behind in an \adhoc{} way.

\end{description}


\begin{figure}
\[
\begin{array}{rcl}
 \mathit{base} & = & \konst{bool} \mid \konst{char} \mid \konst{u8} \mid
 \konst{u16} \mid \konst{u32} \mid \konst{u64}  \mid \konst{i16} \mid
 \konst{i32} \mid \konst{i64} \mid \konst{float} \mid \konst{double} \\
 \tau & = & \mathit{base} \\
      & \mid & \konst{Recd}\; (f_1 : \tau_1) \ldots (f_n : \tau_n) \\
      & \mid & \konst{Array}\; \tau \; \mathit{exp} \\
      & \mid & \konst{Union}\; (\mathit{bexp}_1 : \tau_1) \ldots (\mathit{bexp}_n : \tau_n)
\end{array}
\]
\caption{Contiguity types}
\label{ctypes}
\end{figure}

In the following, we describe a few ideas. Some are straightforward
but necessary building blocks, while others represent significant
advances.

\section{Generalizing contiguity types}

In the current incarnation of contiguity types all data is bounded,
since all \konst{Array} types are given an explicit bound, and all
base types are of fixed sizes. Removing these two restrictions would
greatly increase expressiveness. We have thus been thinking about
adding a lexer and a suitably adapted version of Kleene star. The
augmented syntax can be seen in Figure \ref{gen-contig-types}.

\subsection{Lexing}

Currently, the set of base contiguity types is set to be the usual
base types expected in most programming languages, see Figure
\ref{ctypes}. Semantically, a base type denotes a set of strings of
the specified width, but it is also coupled with an
\emph{interpretation function} for example, the contiguity type
\verb+u8+ denotes the set of all one-byte strings, interpreted by the
usual unsigned valuation function:

  \[ \konst{u8} = \langle \set{s \mid \konst{length}(s) = 1}, \konst{uvalFn} \rangle
  \]

  This cannot, however, capture base types such as string literals of
  arbitrary size, or bignums, or the situation in packed
  bit-level encodings where fields are of \emph{ad hoc} sizes aimed at
  saving space. An obvious generalization is to express base types via
  regular expressions paired with interpretation functions. Now \konst{u8}
  can be defined as

  \[ \konst{u8} = \langle . \; , \konst{uvalFn} \rangle
  \]

\noindent (where `.' is the standard regular expression denoting any character). Similarly,

  \[ \konst{Cstring} = \langle [\backslash 001-\backslash 255]+ \backslash 000, \konst{I} \rangle
  \]

\noindent denotes a zero-terminated string as used in the C
language. Its interpretation is just the identity function, but could
also be a function that dropped the terminating \verb+\000+ character.

\begin{remark} [verification] There already exists a
  HOL4 theory of regexp based, maximal munch, lexer generation,
  and it should be relatively easy to adapt contiguity types to use
  its lexemes instead of the current restricted set of base types. One
  piece of important work yet to be done will be to make the lexer DFA
  be table-driven.
\end{remark}

\subsection{Kleene Star}
  Contiguity types lack Kleene star. The use of bounded \konst{Array}
  types provides much expressiveness for representing sequences of
  data, but ultimately some kinds of message can't be handled, \ie,
  those where there is no way to predict the number of nestings of
  structure. Examples are s-expressions, logical formulas, and
  programming language syntax trees. We address this by providing a
  new contiguity type constructor--- \konst{List} --- of unbounded
  lists. A message matching a $\konst{List}\;\tau$ type will have an
  encoding similar to implementations of lists in functional
  languages. The recognition and parsing algorithms for contiguity
  types are extended to handle \konst{List} objects by iteratively
  unrolling the recursive equation

\[ \kstar{L} = \varepsilon \cup L \cdot \kstar{L} \]

Further work is needed to establish the details, but the following
sketch should give a sense of how we expect things to be.

\begin{figure}
\[
\begin{array}{rcl}
 \tau & =    & \konst{Base}\; \langle \mathit{regexp}, \mathit{valFn} \rangle \\
      & \mid & \konst{Recd}\; (f_1 : \tau_1) \ldots (f_n : \tau_n) \\
      & \mid & \konst{List}\; \tau \\
      & \mid & \konst{Array}\; \tau \; \mathit{exp} \\
      & \mid & \konst{Union}\; (\mathit{bexp}_1 : \tau_1) \ldots (\mathit{bexp}_n : \tau_n)
\end{array}
\]
\caption{Generalized contiguity types}
\label{gen-contig-types}
\end{figure}

\begin{example}[Recursive lists]

\end{example}
The type $\konst{List}\;\tau$ is represented by the following
contiguity type, a recursive record:

\[
 \konst{List}\;\tau =
   \left\{
     \begin{array}{lcl}
       \konst{tag} & : & \konst{u8} \\
       \konst{test} & : &
       \begin{array}[t] {lcl}
         \konst{Union} \{ \\
         \quad \konst{tag} = \konst{NilTag} & \longrightarrow & \varepsilon \\
         \quad \konst{tag} = \konst{ConsTag} & \longrightarrow &
          \{ \konst{hd} : \tau, \ \konst{tl} : \konst{List}\; \tau \}
        \end{array}
     \end{array}
   \right.
\]

In words, a $\konst{List}\;\tau$ matches a sequence of records where
a single-byte tag (\konst{NilTag} or \konst{ConsTag}) is read, then tested to see
whether to stop parsing the list (\konst{NilTag}) or to continue on to
parse a $\tau$ into the \konst{hd} field and recurse in order to
process the remainder of the list. Thus, the list of integers

\[ \konst{Cons}(1, \konst{Cons}(2, \konst{Cons}(3, \konst{Nil}))) \]

can be represented in a message as (\konst{Code} is an encoder for integers)

\[ \konst{ConsTag}\cdot \konst{Code}(1) \cdot
   \konst{ConsTag}\cdot \konst{Code}(2) \cdot
   \konst{ConsTag}\cdot \konst{Code}(3) \cdot \konst{NilTag} \]

and a contig-based parser, given type $\konst{List}\; \konst{int}$
would succeed, returning the context

\[
\begin{array}{rcl}
\konst{root.tag} & \mapsto & \konst{ConsTag} \\
\konst{root.hd} & \mapsto & \konst{Code}(1) \\
\konst{root.tl.tag} & \mapsto & \konst{ConsTag} \\
\konst{root.tl.hd} & \mapsto & \konst{Code}(2) \\
\konst{root.tl.tl.tag} & \mapsto & \konst{ConsTag} \\
\konst{root.tl.tl.hd} & \mapsto & \konst{Code}(3) \\
\konst{root.tl.tl.tl.tag} & \mapsto & \konst{NilTag}
\end{array}
\]

We expect this solution to be compositional, in the sense that
\konst{List} types can be the arguments of other contiguity types, can
be applied to themselves, \eg, $\konst{List}(\konst{List}\;\tau)$,
\etc \; Thus arbitrary branching structures of arbitrary depth can be
specified and parsed with this extension.

\begin{remark} [verification] There already exists a
  HOL4 theory of contiguity types, with a syntax, semantics, and
  parser generator correctness proof. Adding Kleene star to the syntax
  and semantics is straightforwad, and we have outlined above how the
  parser generator would handle recursive lists. The potentially
  unbounded iteration of Kleene star has often posed termination
  problems when formalizing matchers, but we have previous
  experience in circumventing such problems and, moreover, our solution
  above always moves forward in the input when handling a
  $\konst{List}\;\tau$ type.
\end{remark}


%% An important consideration is that---at least in the
%% embedded system context of CASE---messages are essentially flattened
%% versions of datastructures. Such formats are unambiguous and easy to
%% parse. This allows us to ignore some aspects of context-free parsing
%% such as infixity, associativity, and precedence. Implementations will
%% likely require a stack for parsing tree-shaped data, but the full
%% expressive power permitted by context-free grammars and parser
%% generators may not be needed, and may actually be too much.

\section{Context-free parsing}

The outlined approach seems to indicate that the contiguity-type
approach can be extended to certain classes of
\emph{context-free-like} languages.\footnote{It isn't clear where in
  the Chomsky hierarchy this work falls.} However, the approach is
distinctly different in flavor, mainly because sums are determined by
\emph{looking behind} when computing which choice to follow in a
\konst{Union} type. Thus the list parser sketched above branches
\emph{after} it has seen the tag. This approach runs into difficulties
when confronted by grammars that use non-deterministic sum to express
branching. Consider the following grammar rules for JSON-style lists
(called \konst{Array} in JSON specifications). We assume the lexer has
removed whitespace and note that occurrences of open and close
brackets in the rules are terminals of the grammar.
\[
\begin{array}{rcl}
 \konst{Array} & ::= & [\; ] \mid [ \konst{Elts} ] \\
 \konst{Elts}  & ::= & \konst{Value} \mid \konst{Value} ,  \konst{Elts} \\
\end{array}
\]

Without diving too deeply into details, a parser for this language has
to deal with a situation where it has read the initial open bracket
([) for an \konst{Array} and now needs to parse either a closing
  bracket (for the empty list) or a \konst{Value}. In this case, the
  parser needs lookahead to resolve the choice, and the look-behind of
  contiguity types isn't helpful. It could be that the difficulty can
  be avoided, by somehow pretending that the next lexeme is in the
  context, and thus available in the evaluation of boolean expressions
  of the \konst{Union} construct. But that raises other questions
  which are currently being worked on. It is likely that such
  questions find answers in the voluminous literature on parsing, \eg,
  the computation of ``first'' and ``follow'' sets in parser
  generators may be pertinent. It would, in any case, be a nice result
  if there was a smooth combination of look-behind and look-ahead
  parsing.

\subsection{Contiguity types as lexemes}

Standard parser technology is based on a set of grammar rules phrased
in terms of a collection of terminals and non-terminals, where the
terminals are specified by regular expressions. It seems interesting
to consider specifying terminals by contiguity types rather than
regular expressions. Thus, suppose we have a contiguity type
specifying wellformed GPS coordinates

{\small
\begin{verbatim}
  gps = {lat : double
         lon : double
         alt : double
         check : Assert ...
        }
\end{verbatim}}

and wanted to specify a list of such. Then the following grammar rules
seem natural in parsing a binary encoded list of wellformed GPS
coordinates:
\[
\begin{array}{rcl}
 \konst{GPSList} & ::=  & \konst{NilTag} \\
                 & \mid & \konst{ConsTag} \cdot \konst{gps} \cdot \konst{GPSList} \\
\end{array}
\]

The lexeme \konst{gps} is produced by the contiguity type parser, as
are the tags, while the sequencing of terminals and non-terminals,
along with the recursion, is handled by the grammar rules. The
construction of parse trees from the grammar must smoothly incorporate
parse trees coming from the lexer. This general approach would
provide, if the implementation can be made to work, a nice example of
combining contiguity types and standard parser technology, while
leaving both implementations mostly unchanged. However, we expect
there are quite a few obstacles to be surmounted.

\begin{remark} [verification] There already exists
  HOL4 theories of SLR and PEG parsing. Similarly, there are several
  verified parser generators in the Coq world. It would be interesting
  to marry that work with a verified contiguity type lexer to get a
  fully verified parser generator implementation with builtin
  assertion checking.
\end{remark}

\subsection{Stratified (modular) parsing}

 One common requirement in parsing is modularity: \eg, a parser for
 programming-language statements is notionally parameterized by a
 parser for expressions. In most cases, these languages can be
 combined in the same grammar, but that works against worthy aims such
 as re-use and separation. The SafeDocs experience shows that this
 idea must be taken seriously for difficult languages such as
 PDF. What seems to be needed is a way to---in the midst of parsing
 language A---select a subsequence of the input to separately parse
 with parser B. In determining the slice of the input to give to B,
 the accumulated context often needs to be consulted. Contiguity types
 give a way to handle this, and, in particular, seem to offer a
 framework in which the context accumulated by B is added back to that
 of A. Message processing, with its piggy-backing of messages, has
 much the same flavor (albeit simpler).

\section{Efficient implementations}

\section{Examples}

It is worth having some simple but challenging examples in mind.

\begin{enumerate}

\item A JSON parser that enforces the requirement that \textsf{Object}
  elements\footnote{An \textsf{Object} is essentially a finite map
    where the domain is strings.} have no duplicate keys. It is of
  course quite easy to enforce this property on the resulting parse
  tree, by making another pass over the data, but the challenge is to
  combine parsing and data validation in a single pass. How can we
  express this constraint inside the grammar, and generate an
  implementation that checks the constraint on the fly? Note that the
  messages in this example do not have a self-describing aspect.

\item A filter for wellformed first order terms where, to be
  \emph{wellformed}, all functions need to be applied to the correct number
  of arguments. Thus message input for the filter would have the form

  \[ \langle \mathit{sig} , \mathit{term}\; \konst{list} \rangle \]

  and, given a message $\langle S, L \rangle$, each term $t$ in $L$
  needs to be (recursively) checked to ensure that each application
  $f(t_1,\ldots,t_n)$ is such that $f$ is in $S$, with arity $n$.
  (Note there is another constraint, namely that $S$ has no duplicate
  function names, as in (1).)

\item Embedding the wellformed terms of (2) into the JSON parser of
  (1) in order to get a parser for wellformed terms encoded in
  JSON. This example captures a common pattern, for which JSON and XML
  for example are well-suited, namely using a general format to
  represent a specific language.

\end{enumerate}

\section{Conclusion}

The world of parsing, one of the great achievements of Computer
Science, continues to generate new research problems. The ideas
sketched above, if achievable, can solve some problems not addressed
by current technology.



%% \section{Property-enhanced lexing}

%%    We have a formalized, proved correct lexer generator that uses the
%%    maximal-munch heuristic. To be more performant, it needs to be
%%    converted to be table-driven, and we would like it to be a
%%    property-producing lexer.

%%    Conventional lexers return a stream of lexemes. Regular expressions
%%    are able to enforce basic well-formedness properties of lexemes,
%%    and these would need to be propagated to the parser in order for
%%    the parser to enforce its own well-formedness constraints. This can
%%    be extended to stratified parsing situations (see below).


%% \paragraph{Contiguity types and lexing}

%%    Two ideas seem worth thinking about a little more. In one, we use
%%    contiguity types as a high-powered lexer language; in the other, we
%%    add a regexp-based lexer as a base contiguity type.

%% \begin{itemize}

%% \item Since contiguity types are deterministic, they could possibly be
%%       useful as a lexer replacement in certain parsing scenarios. A
%%       contig-type parser being used as a lexer would generate a "parse
%%       tree" lexeme that would be folded into the parser stack.

%% \item Contiguity types with regexp-based lexing added as a supplement
%%       to the existing suite of fixed-width base types. This amounts to
%%       a controlled use of Kleene star. This would add a nice way to
%%       handle C-style strings in messages, or any other such message
%%       portion of arbitrary length ending with a specific delimiter.

%% \end{itemize}

\end{document}
