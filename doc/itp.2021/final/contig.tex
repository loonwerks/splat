\documentclass[a4paper,UKenglish,cleveref, autoref, thm-restate]{lipics-v2021}

\bibliographystyle{plainurl}% the mandatory bibstyle

\title{Specifying Message Formats with Contiguity Types} %TODO Please add
% \titlerunning{Dummy short title} %TODO optional, please use if title is longer than one line

\author{Konrad Slind}{Trusted Systems Group, Collins Aerospace, USA}{konrad.slind@collins.com}{}{}
\authorrunning{K. Slind}
\Copyright{Konrad Slind}
\ccsdesc[500]{Theory of computation~Logic and verification}
\keywords{Logic, verification, formal language theory, message format languages}
\acknowledgements{This research was developed with funding from the
  Defense Advanced Research Projects Agency (DARPA). The views,
  opinions and/or findings expressed are those of the author and
  should not be interpreted as representing the official views or
  policies of the Department of Defense or the
  U.S. Government. Discussions with Robby, Luke Ryon, Dave Greve,
  David Hardin, Magnus Myreen, and Michael Norrish have greatly
  improved this work.}

% Artifact URL
\supplement{HOL4 formalization may be found at:}
\supplementdetails{Model} {https://github.com/loonwerks/splat/blob/monitors/contig/abscontigScript.sml}

\nolinenumbers %uncomment to disable line numbering

%\hideLIPIcs  %uncomment to remove references to LIPIcs series (logo, DOI, ...), e.g. when preparing a pre-final version to be uploaded to arXiv or another public repository

%% %Editor-only macros:: begin (do not touch as author)%%%%%%%%%%%%%%%%%%%%%%%%%%%%%%%%%%
%% \EventEditors{John Q. Open and Joan R. Access}
%% \EventNoEds{2}
%% \EventLongTitle{42nd Conference on Very Important Topics (CVIT 2016)}
%% \EventShortTitle{CVIT 2016}
%% \EventAcronym{CVIT}
%% \EventYear{2016}
%% \EventDate{December 24--27, 2016}
%% \EventLocation{Little Whinging, United Kingdom}
%% \EventLogo{}
%% \SeriesVolume{42}
%% \ArticleNo{23}
%% %%%%%%%%%%%%%%%%%%%%%%%%%%%%%%%%%%%%%%%%%%%%%%%%%%%%%%

% Packages and abbreviations used by Konrad
\usepackage{latexsym}
\usepackage{amsmath}
\usepackage{amssymb}
\usepackage{amsbsy}

\newcommand{\ie}{\textit{i.e.}}
\newcommand{\eg}{\textit{e.g.}}
\newcommand{\etal}{\textit{et al.}}
\newcommand{\etc}{\textit{etc.}}
\newcommand{\adhoc}{\textit{ad hoc}}
\newcommand{\konst}[1]{\ensuremath{\mathsf{#1}}}
\newcommand{\imp}{\Rightarrow}
\newcommand{\lval}{\ensuremath{\mathit{lval}}}
\newcommand{\set}[1]{\ensuremath{\{ {#1} \}}}
\newcommand{\kstar}[1]{\ensuremath{{#1}^{*}}}
\newcommand{\Lang}[1]{\ensuremath{{\mathcal L}({#1})}}
\newcommand{\LangTheta}[1]{\ensuremath{{\mathcal L}_{\theta}({#1})}}
\newcommand{\SubstTheta}[2]{\ensuremath{\theta\;{#1}\; {#2}}}
\newcommand{\itelse}[3]{\mbox{$\mathtt{if}\ {#1}\ \mathtt{then}\ {#2}\ \mathtt{else}\ {#3}$}}

\EventEditors{Liron Cohen and Cezary Kaliszyk}
\EventNoEds{2}
\EventLongTitle{12th International Conference on Interactive Theorem Proving (ITP 2021)}
\EventShortTitle{ITP 2021}
\EventAcronym{ITP}
\EventYear{2021}
\EventDate{June 29--July 1, 2021}
\EventLocation{Rome, Italy (Virtual Conference)}
\EventLogo{}
\SeriesVolume{193}
\ArticleNo{27}

\begin{document}

\maketitle


\begin{abstract}
We introduce \emph{Contiguity Types}, a formalism for network message
formats, aimed especially at self-describing formats. Contiguity types
provide an intermediate layer between programming language data
structures and messages, offering a helpful setting from which to
automatically generate decoders, filters, and message generators.  The
syntax and semantics of contiguity types are defined and used to prove
the correctness of a matching algorithm which has the flavour of a
parser generator. The matcher has been used to enforce semantic
well-formedness conditions on complex message formats for an
autonomous unmanned avionics system.
\end{abstract}


\section{Introduction}\label{sec:intro}

Serialized data, for example network messages, is an important
component in many computer systems.\footnote{DISTRIBUTION STATEMENT
  A. Approved for public release.}  As a result, innumerable libraries
and tools have been created that use high level specifications as a
basis for automating the creation, validation, and decoding of such
data. Usually, these high level specifications describe the format of
a message in terms of how the elements (fields) of the message are
packed side-by-side to make the full message. When the size of each
field is known in advance, there are really no conceptual
difficulties. However, messages can be more complicated than that.

The main source of difficulty is \emph{self-describing} messages:
those where information embedded in fields of the message determines
the final structure of the message. Two of the main culprits are
variable-length arrays and unions. A \emph{variable-length array} is a
field where the number of elements in the field depends on the value
of some already-seen field (or, more generally, is the result of a
computation involving previously-seen information in the message).
The length is therefore a value determined at runtime. A \emph{union}
is deployed when some information held in a message is used to
determine the structure of later portions of the message. For example,
unions can be used to support versioning where version $i$ has $n$
fields, and version $i+1$ has $n+1$. In settings where both versions
need to be supported in a single format, it can make sense to encode
the version handling inside the message, and unions are how this can
be specified.

We believe that tools and techniques from formal language theory such
as regular expressions, automata, grammars, parser generators, \etc\,
can provide an effective way to tackle message formats, and have been
using the acronym \emph{SPLAT} (Semantic Properties for Language and
Automata Theory) to refer to this approach. For example, we have used
regular expressions as a specification language for message formats
having simple interval constraints on the values allowed in
fields. Generation of the corresponding DFA results in an efficient
table-driven automaton implementing the specified constraints, with a
solid proof certificate connecting the original constraints with the
DFA behavior \cite{hardin-slind-safecomp-2016}.

However, self-describing data formats fall outside the realm of common
formal language techniques; \eg, variable-length fields are clearly
not able to be described by regular or context-free languages. (These
language classes encompass repetitions of a fixed or unbounded size,
but not repetitions of a size determined by parts of the input
string.) It seems that context-sensitive grammars can, in principle,
specify such information, but there are few tools supporting context
sensitive languages. Knuth introduced attribute grammars
\cite{knuth-attribute-grammars} for dealing with context-sensitive
aspects of parsing, and those techniques address similar problems to
ours. Another possibility would be to use \emph{parser combinators} in
order to quickly stitch together a parser; it seems likely that the
combinators can be instrumented to gather and propagate contextual
information. However, we are seeking a high level of formal
specification and automation, while still being rooted in formal
languages, with their emphasis on sets of strings as the basic notion.

\section {Contiguity Types}

The characteristic property of a message is \emph{contiguity}: all the
elements of the message are laid out side-by-side in a byte array (or
string). Our assumption is that a message is the \emph{result} of a
map from structured data and we will rely on a basic collection of
programming language types to capture that structure. Contiguity types
(Figure \ref{contig-types}) start with common base types (booleans,
characters, signed and unsigned integers, \etc) and are closed under
the construction of records, arrays, and unions. \footnote{We will use
the terms ``\emph{contiguity type}, \konst{contig}, and $\tau$
interchangeably.}

\begin{figure}
\[
\begin{array}{rcl}
 \mathit{base} & = & \konst{bool} \mid \konst{char} \mid \konst{u8} \mid
 \konst{u16} \mid \konst{u32} \mid \konst{u64}  \mid \konst{i16} \mid
 \konst{i32} \mid \konst{i64} \mid \konst{float} \mid \konst{double} \\
 \tau & = & \mathit{base} \\
      & \mid & \konst{Recd}\; (f_1 : \tau_1) \ldots (f_n : \tau_n) \\
      & \mid & \konst{Array}\; \tau \; \mathit{exp} \\
      & \mid & \konst{Alt}\; \mathit{bexp}\; \tau \; \tau
\end{array}
\]
\caption{Contiguity types}
\label{contig-types}
\end{figure}

Notice that $\tau$ is defined in terms of a type of arithmetic
expressions $\mathit{exp}$ and also $\mathit{bexp}$, boolean
expressions built from $\mathit{exp}$. Now consider
%
\[
 \konst{Array} \; \tau \; \mathit{exp} \ .
\]
%
For this to specify a varying length array dependent on other fields
of the message, its dimension $\mathit{exp}$ should be able to refer
to the \emph{values} of those fields. The challenge is just how to
express the concept of ``other fields'', \ie, we need a notation to
describe the \emph{location} in the message buffer where the value of
a field can be accessed. Our core insight is that this is similar to a
problem that programming language designers had in the 60s and 70s,
resolved by the notions of \emph{L-value} and \emph{R-value}. The idea
is originally due to Christopher Strachey in CPL \cite{strachey} and
developed subsequently, for example by Dennis Ritchie in
C \cite{dmr-lvals}.

Before getting into formal details, we discuss a few examples.  We
will use familiar notation: records are lists of $\mathit{fieldname} :
\tau$ elements enclosed by braces; an array field
$\konst{Array}\;c\;\mathit{dim}$ is written $c\, [\mathit{dim}]$; and
$\konst{Alt}\;b\;\tau_1\;\tau_2$ is written
`$\itelse{b}{\tau_1}{\tau_2}$'. `Cascaded' \konst{Alt}s may be written
in Lisp `cond' style, \ie, as
\[
\begin{array}{ll}
\konst{Alt} & b_1 \longrightarrow \tau_1 \ldots \\
            & b_n \longrightarrow \tau_n \\
            & \text{otherwise} \longrightarrow \tau_{n+1}
\end{array}
\]

\begin{enumerate}

\item The following is a record with no self-describing aspects: each
  field is of a statically known size.

{\small
\begin{verbatim}
  {A : u8
   B : {name : char [13]
        cell : i32}
   C : bool
  }
\end{verbatim}
}
The \verb+A+ field is specified to be an unsigned int of width 8 bits,
the \verb+B+ field is a record, the first element of which is a
character array of size 13, and the second element of which is a 32
bit integer; the last field is specified to be a
boolean.

\item Variable-sized strings are a classic self-describing aspect. In
  this example the contents of the \verb+len+ field determines the
  number of elements in the \verb+elts+ field.

{\small
\begin{verbatim}
  { len : u16
    elts : char [len]
  }
\end{verbatim}
}

\item The following example shows the \konst{Alt} construct being
  used to support multiple versions in a single format.  Messages with
  the value of field \verb+versionID+ being less than 14 have three
  fields in the message, while all others have two.

{\small
\begin{verbatim}
  {versionID : u8
   versions : if versionID < 14 then
                 { A : i32, B : u16})
              else { Vec : char [13]}
   }
\end{verbatim}
}

\item The following is a contrived example showing the need for
  resolution of multiple similarly named fields; it also shows how the
  information needed to determine the message structure may be deeply
  buried in some fields.
{\small
\begin{verbatim}
  {len : u16
   A : {len : u16
        elts : u16[len]
       }
   B : char [A.len - 1 * len]
   C : i32 [A.elts[0]]
  }
\end{verbatim}
}
\end{enumerate}

\subsection{Expressions, L-values, and R-values}

In programming languages, an \emph{L-value} is an expression that can
occur on the left-hand side of an assignment statement. Similarly, an
\emph{R-value} can occur on the right-hand side of
assignments. Following are a few examples:
{\small
\begin{verbatim}
    x := x + 1
    A[x] := B.y + 42
    A[x].lens.fst[7] := MAX_LEN * 1024 + B.y
\end{verbatim}
}
Figure \ref{Lvalues} presents the formal syntax for L-values, R-values,
and the boolean expressions we will use.  An L-value can be
a variable, an array index, or a record field access. R-values are
arithmetic expressions that can contain L-values (we will use
$\mathit{exp}$ interchangeably with R-value).

\begin{figure}
\[
\begin{array}{rcl}
\mathit{lval} & = & \mathit{varname} \mid
                    \mathit{lval} \, [ \mathit{exp} ] \mid
                    \mathit{lval} . \mathit{fieldname} \\
%  & & \\
\mathit{exp} & = & \konst{Loc}\; \mathit{lval}
              \mid \konst{nLit}\; \konst{nat}
              \mid \mathit{constname}
              \mid \mathit{exp} + \mathit{exp}
              \mid \mathit{exp} * \mathit{exp} \\
%  & & \\
\mathit{bexp} & = & \konst{bLoc}\; \mathit{lval}
              \mid  \konst{bLit}\; \konst{bool}
              \mid  \neg \mathit{bexp}
              \mid  \mathit{bexp} \land \mathit{bexp}
              \mid  \mathit{exp} = \mathit{exp}
              \mid  \mathit{exp} < \mathit{exp}
\end{array}
\]
\caption{L-values, expressions, and boolean expressions}
\label{Lvalues}
\end{figure}

An L-value denotes an \emph{offset} from the beginning of a
data structure, plus a \emph{width}. In an R-value, an occurrence of an
L-value is mapped to the value of the patch of memory between
$\mathit{offset}$ and $\mathit{offset} + \mathit{width}$. For the
purpose of specifying message formats, it may not be immediately
obvious that a notation supporting assignment in imperative languages
can help, but there is indeed a form of assignment lurking.

The above explanation of L-values centers on indices into a byte
buffer; in the following we will give a mild variant of this: instead
of indices into the buffer, we lift out the designated slices. Thus,
given environments $\theta: \mathit{lval} \mapsto \konst{string}$ (binding
L-values to strings), and functions
$\konst{toN}:\konst{string}\to\mathbb{N}$ and
$\konst{toB}:\konst{string}\to\konst{bool}$ (which interpret byte
sequences to numbers and booleans, respectively), expression
evaluation and boolean expression evaluation have conventional
definitions:

\[
\begin{array}{l}
\konst{evalExp} \;\theta \; e =
\mathtt{case}\; e\
 \left\{
 \begin{array}{lcl}
    \konst{Loc}\; \lval & \Rightarrow & \konst{toN}(\theta(\lval)) \\
    \konst{nLit}\; n & \Rightarrow & n  \\
    e_1 + e_2 & \Rightarrow & \konst{evalExp}\;\theta\; e_1 + \konst{evalExp}\;\theta\; e_2  \\
    e_1 * e_2 & \Rightarrow & \konst{evalExp}\;\theta\; e_1 * \konst{evalExp}\;\theta\; e_2  \\
  \end{array}
 \right.
 \\ \\
\konst{evalBexp}\;\theta \; b =
\mathtt{case}\; b\
 \left\{
 \begin{array}{lcl}
    \konst{bLoc}\; \lval & \Rightarrow & \konst{toB}(\theta(\lval)) \\
    \konst{bLit}\; b & \Rightarrow & b \\
    \neg b & \Rightarrow & \neg(\konst{evalBexp}\;\theta \; b)  \\
    b_1 \lor b_2 & \Rightarrow & \konst{evalBexp}\;\theta \;b_1 \lor \konst{evalBexp}\;\theta \;b_2   \\
    b_1 \land b_2 & \Rightarrow & \konst{evalBexp}\;\theta \;b_1 \land \konst{evalBexp}\;\theta \;b_2   \\
    e_1 = e_2 & \Rightarrow & \konst{evalExp}\;\theta \;e_1 = \konst{evalExp}\;\theta \;e_2   \\
    e_1 < e_2 & \Rightarrow & \konst{evalExp}\;\theta \;e_1 < \konst{evalExp}\;\theta \;e_2
  \end{array}
 \right.

\end{array}
\]

\begin{remark}[Partiality]
Expression evaluation is partial because there is no guarantee that
$\theta(\lval)$ is defined: an \lval{} being looked-up may not be in the
map $\theta$. Failure in evaluation is modelled by the \konst{option}
type, and must be handled in the semantics and the matching
algorithm. However error handling is omitted in the presentation since
it hampers readability. See the HOL4 formalization for full details.
\end{remark}

\subsection{Semantics}

 We now confess to misleading the reader: in spite of the notational
 similarity, a contiguity type is \emph{not} a type: it is a formal
 language. A type is usually understood to represent a set, or domain,
 of values, \eg, the type \konst{int32} represents a set of
 integers. In contrast, the contiguity type \konst{i32} represents the
 set of strings of width 32 bits. An element of a contiguity type can
 be turned into an element of a type by providing interpretations for
 all the strings at the leaves and interpreting the \konst{Recd} and
 \konst{Array} constructors into the corresponding type constructs. (A
 base contiguity type therefore serves mainly as a \emph{tag} to be
 interpreted as a width and also as an intended target type.) Thus,
 contiguity types sit---conveniently---between the types in a
 programming language and the strings used to make messages.

The semantics definition depends on a few basic notions familiar from
language theory: language concatenation, and iterated language
concatenation.

\begin{align*}
L_1 \cdot L_2 &= \set{w_1 w_2 \mid w_1 \in L_1 \land w_2 \in  L_2} \\
L^0 &= \varepsilon \\
L^{n+1} &= L \cdot L^n
\end{align*}


\begin{definition}[Semantics of contiguity types]

 In the following, we assume given an assignment $\theta$ for
 evaluating expressions. If an expression evaluation fails, the
 language being constructed will be $\emptyset$.

\[
% \begin{array}{l}
\LangTheta{\tau} =
\mathtt{case}\; \tau\
% \hspace*{3mm}
 \left\{
 \begin{array}{l}
 \mathit{base} \Rightarrow \set{s \mid \konst{len}(s) = \konst{width}(base)} \\
 \konst{Recd}\; (f_1 : \tau_1) \ldots (f_n : \tau_n)
      \Rightarrow \LangTheta{\tau_1} \cdot \ldots \cdot \LangTheta{\tau_n}
\\
 \konst{Array}\; \tau \; \mathit{exp} \Rightarrow  \\
  \hspace*{5mm}
 \left\{
 \begin{array}{ll}
    \LangTheta{\tau}^{\konst{evalExp}\;\theta\;\mathit{exp}} &
       \mathrm{if}\ \konst{evalExp}\;\theta\;\mathit{exp}\ \mathrm{succeeds} \\
    \emptyset & \text{if evaluation fails}
 \end{array}
 \right.
\\
 \konst{Alt}\; \mathit{bexp}\;\tau_1\; \tau_2 \Rightarrow \\
  \hspace*{5mm}
 \left\{
 \begin{array}{ll}
    \LangTheta{\tau_1} & \mathrm{if}\ \konst{evalBexp}\;\theta\;\mathit{bexp} = \konst{true} \\
    \LangTheta{\tau_2} & \mathrm{if}\ \konst{evalBexp}\;\theta\;\mathit{bexp} = \konst{false} \\
    \emptyset          & \text{if evaluation fails}
 \end{array}
 \right.
 \\
\end{array}
 \right.
%\end{array}
\]
\end{definition}

\begin{example}
Consider the following schema for an \emph{option} contiguity
type. The empty record \verb+{}+ associated with boolean expression
$b$ has no fields.
\[ \itelse{b}{\{\,\}}{c} \]
In case $\mathit{b}$ evaluates to \konst{true}, no portion of the
string is consumed; otherwise, $c$ specifies the remainder of the
processing. It may be instructive to consider how this type works with
arrays. For example, a string meeting the following \konst{contig} specification

{\small
\begin{verbatim}
  (if b then {} else i32) [3]
\end{verbatim}
}

\noindent is either zero or twelve bytes in length (assuming that
\verb+i32+ is four bytes wide).

\end{example}

\section{Algorithms}

The following are classical topics in formal language theory and
practice, and they are worth investigating in the context of
contiguity types. At present we have been working on decoding,
filtering, and test generation.

\begin{description}

\item [Decoding] A decoder breaks a sequence of bytes up and puts the
  pieces into a useful data structure, typically a parse tree. We will
  discuss this in more detail in Section \ref{decoding}.

\item [Filtering] A filter computes an answer to the question: ``does
  a sequence of bytes meet the specification of a given contiguity
  type''. This is an instance of the language recognition
  problem. More powerful filters enforce that certain fields of a
  message, when interpreted, meet specific semantic properties. We
  will discuss this further in Section \ref{assert}.

\item [Serialization] Given a contiguity type, synthesize a function
  that writes a compact binary version of a data structure to a message.

\item [Test generation] Given a contiguity type, generate byte
  sequences that do (or do not) meet its specification and feed the
  sequences to implementations in order to observe their behaviour.

\item [Learning] Given training sets of messages that are
  accepted/rejected by an implementation, attempt to discover a
  contiguity type for the entire set of messages.

\end{description}

\subsection{Decoding}
\label{decoding}

Above we mentioned that decoding can result in parse trees; however,
self-describing messages allow a different conceptual framework to be
brought to bear. There is an important distinction between parsing,
which \emph{creates} structure (parse trees), and matching, which is
given structure and calculates assignments
(substitutions).\footnote{Thus the notion of matching discussed here
  is in the tradition of term rewriting \cite{baader:nipkow}, the main
  difference being that our substitutions are applied to
  $\mathit{lval}$s rather than variables.} Giving some evocative types
helps make the difference clear:
%
\begin{align*}
  \konst{parse} &: \mathit{grammar} \to \konst{string} \to \mathit{parse tree} \\
  \konst{match} &: \mathit{pattern} \to \konst{string} \to \mathit{assignments}
\end{align*}

For our purposes, namely decoding datastructures in binary format, the
central decoding algorithm is a \emph{matcher}: given a contiguity
type $\tau$ and a string $s$, the matcher will either fail, or succeed
with an assignment $\theta : \mathit{lval} \mapsto \konst{string}$
mapping each L-value in $\tau$ to its corresponding slice of $s$. The
assignment $\theta$ can be post-processed to yield a standard parse
tree, but its novelty and strength is that $\theta$ can be dynamically
consulted to access the values needed to guide the processing of
self-describing messages.

\begin{definition}[Matching algorithm]

The matcher operates over a triple
$(\mathit{worklist},\mathit{str}, \theta)$ where $\mathit{worklist}$ is a stack used to
linearize the input contiguity type $\tau$, $\mathit{str}$ represents
the remainder of the input string, and $\theta$ is the assignment
being built up. Each element of the $\mathit{worklist}$ is a
$(\tau,\mathit{lval})$ pair, where $\tau$ is a \konst{contig}, and
$\mathit{lval}$ is the path growing down from the root to $\tau$. The
notation $(\lval \mapsto \mathit{slice}) \bullet \theta$ denotes the
addition of binding $\lval \mapsto \mathit{slice}$ to $\theta$.
We examine the cases in turn:

\begin{enumerate}

\item The worklist is empty; the match succeeds.
%
\begin{align*}
([], \mathit{str}, \theta) &\Rightarrow \konst{SOME}(\mathit{str}, \theta)
\end{align*}

\item The first element of the worklist is a base type. The
  prescribed number of bytes are broken off the front of the string, giving
  $\mathit{str} = (\mathit{slice},\mathit{rst})$; then the
  binding is added to $\theta$ before recursing. If the string is shorter
  than the requested number of bytes, fail.
%
\begin{align*}
((\konst{Basic}\;a, \lval)::t, \mathit{str}, \theta)
   &\Rightarrow
  (t,\mathit{rst}, (\lval \mapsto \mathit{slice}) \bullet \theta)
\end{align*}

\item The first element of the worklist is $\mathit{recd} =
  \konst{Recd}\;(f_1 : \tau_1) \ldots (f_n : \tau_n)$. Before
  recursing, the fields are pushed onto the stack, extending the path
  to each field element:
%
\begin{align*}
((\mathit{recd}, \lval)::t, \mathit{str}, \theta)
   &\Rightarrow
  ([(\tau_1,\lval.f_1), \cdots , (\tau_n,\lval.f_n)] @ t,\mathit{str}, \theta)
\end{align*}

\item The first element of the worklist is an array. The dimension
  expression is evaluated to get the width $d$, then $d$ copies are
  pushed onto the stack, where each path is extended with the array
  index.
%
\begin{align*}
((\konst{Array}\; \tau \; \mathit{exp},\lval)::t, \mathit{str}, \theta)
   &\Rightarrow
  ([(\tau,\lval[0]), \cdots , (\tau,\lval[d-1])] @ t,\mathit{str}, \theta)
\end{align*}

\item The first element of the worklist is an \konst{Alt}. If $b$ evaluates to \konst{true},
  $\tau_1$ is pushed on to the worklist; if it evaluates to \konst{false}, $\tau_2$ is pushed.
  Otherwise, fail.
%
\begin{align*}
((\konst{Alt}\; b\;\tau_1\;\tau_2, \lval)::t, \mathit{str}, \theta)
   &\Rightarrow
  ((\tau_i,\lval)::t,\mathit{str}, \theta)
\end{align*}

\end{enumerate}

\noindent The matcher function, \konst{match} begins with an initial state
%
\[
  \mathit{state}_0 = ([(\mathit{root},\tau)],\mathit{str}_0,\emptyset)
\]

where the initial path is a default \lval{} variable named
$\mathit{root}$, the initial string is $\mathit{str}_0$, and the
initial assignment has no bindings.

\end{definition}

\begin{theorem}[Matcher termination]
As mentioned, the state of the matcher is held in a
$(\mathit{worklist},\mathit{str}, \theta)$ tuple. The termination
relation is a lexicographic combination, where either $\mathit{str}$
gets shorter, or, $\mathit{str}$ is unchanged and $\mathit{worklist}$
gets smaller under the multiset order. (The multiset order is useful
in this proof since the handling of the \konst{Array} construct is a
nice version of the Hercules-Hydra problem \cite{dershowitz:manna}.)
\end{theorem}


\begin{definition}[Substitution application]

Correctness depends on an operation $\SubstTheta{\mathit{lval}}{\tau}$
applying substitution $\theta$ to contiguity type $\tau$, starting
at $\mathit{lval}$, in order to reconstruct the original string.
%
\[
\SubstTheta{\mathit{lval}}{\tau} =
\mathtt{case}\; \tau\
 \left\{
 \begin{array}{l}
 \mathit{base} \Rightarrow \theta(\mathit{lval})
\\
 \konst{Recd}\; (f_1 : \tau_1) \ldots (f_n : \tau_n) \Rightarrow
   \SubstTheta{(\mathit{lval}.f_1)}{\tau_1} \cdot \ldots \cdot \SubstTheta{(\mathit{lval}.f_n)}{\tau_n}
\\
 \konst{Array}\; \tau_1 \; \mathit{exp} \Rightarrow \\
  \hspace*{5mm}
 \left\{
 \begin{array}{ll}
  \SubstTheta{(\mathit{lval}[0])}{\tau_1} \cdot \ldots \cdot
  \SubstTheta{(\mathit{lval}[d-1])}{\tau_1}, &
   \text{if}\; d = \konst{evalExp}\;\theta\;\mathit{exp} \\
    \emptyset & \text{if evaluation fails}
 \end{array}
 \right.
 \\
 \konst{Alt}\; \mathit{bexp}\;\tau_1\; \tau_2 \Rightarrow \\
  \hspace*{5mm}
 \left\{
 \begin{array}{ll}
    \SubstTheta{\mathit{lval}}{\tau_1} & \mathrm{if}\ \konst{evalBexp}\;\theta\;\mathit{bexp} = \konst{true} \\
    \SubstTheta{\mathit{lval}}{\tau_2} & \mathrm{if}\ \konst{evalBexp}\;\theta\;\mathit{bexp} = \konst{false} \\
    \emptyset          & \text{if evaluation fails}
 \end{array}
 \right.
 \\
\end{array}
 \right.
%\end{array}
\]

\end{definition}

\begin{theorem}[Correctness of substitution]
The correctness statement for the matcher is similar to those found in
the term rewriting literature, namely that the computed substitution
applied to the contiguity type yields the original string:
%
\[
  \konst{match}\;\mathit{state}_0 = \konst{SOME}(\theta, s)
  \imp \theta\;\mathit{root}\;\tau \cdot s = \mathit{str}_0
\]
%
\begin{proof}
By induction on the definition of \konst{match}.
\end{proof}
\end{theorem}

\begin{theorem}[Matcher soundness]
The connection to $\LangTheta{\tau}$ is formalized as
%
\[
  \mathit{str}_0 = s_1 s_2 \land \konst{match}\; \mathit{state}_0 =
  \konst{SOME}(\theta, s_2) \imp s_1 \in \LangTheta{\tau}
\]
%
\begin{proof}
By induction on the definition of \konst{match}.
\end{proof}
\end{theorem}

In other words, a successful match provides a $\theta$ that will
successfully evaluate all encountered expressions, and the matched
string is indeed in the language of $\tau$. A completeness theorem
going in the other direction has not yet been tackled.

\begin{example}
Given the \konst{contig}

{\small
\begin{verbatim}
   {A : Bool
    B : Char
    len : u16
    elts : i32 [len]
  }
\end{verbatim}
}
\noindent and an input string (listed in hex)
{\small
\begin{verbatim}
  [0wx1, 0wx67, 0wx0, 0wx5, 0wx0, 0wx0, 0wx0, 0wx19, 0wx0, 0wx0,
   0wx9, 0wx34, 0wx0, 0wx0, 0wx30, 0wx39, 0wx0, 0wx0, 0wxD4,
   0wx31, 0wxFF, 0wxFF, 0wxFE, 0wxB3]
\end{verbatim}
}
\noindent created by encoding: the boolean \verb+true+, the letter \verb+g+, the
number 5 (MSB 2 byte unsigned), and the five MSB 4 byte signed twos complement
integers 25, 2356, 12345, 54321, and -333, the matcher creates the
following assignment of $\mathit{lval}$s to substrings of the input:
{\small
\begin{verbatim}
 [(root.A,       (Bool, [0wx1])),
  (root.B,       (Char, [0wx67])),
  (root.len,     (u16,  [0wx0, 0wx5])),
  (root.elts[0], (i32,  [0wx0, 0wx0, 0wx0, 0wx19])),
  (root.elts[1], (i32,  [0wx0, 0wx0, 0wx9, 0wx34])),
  (root.elts[2], (i32,  [0wx0, 0wx0, 0wx30, 0wx39])),
  (root.elts[3], (i32,  [0wx0, 0wx0, 0wxD4, 0wx31])),
  (root.elts[4], (i32,  [0wxFF, 0wxFF, 0wxFE, 0wxB3]))
 ]
\end{verbatim}
}
Note that each element of the list is of the form
$(\mathit{lval},(\mathit{tag},\mathit{bytes}))$ where each slice is labelled with its
corresponding base type, to support further translation.
\end{example}

Thus the matcher will break up the input string in accordance with the
specification; the execution, in effect, generates a sequence of
assignments that, if applied, would populate a data structure with the
specified data in the specified places. Therefore it is not really
necessary to generate parse trees to in order to decode messages: one
merely needs a target data structure to write data into. (In fact,
when filtering, no target data structure is needed at all.) The
correctness property will ensure that \emph{all} fields are written
with the specified data. The assignments can be incrementally
evaluated as the decoder runs, or can be stored and applied when the
decoder terminates.

%\subsection{Parsing}

\section{Extended contiguity types}

In the discussion so far, contiguity types can only express bounded
data: each base type has a fixed size and all \konst{Array} types are
given an explicit bound. Removing these two restrictions would greatly
increase expressiveness. Of course, we look to the theory of formal
languages to guide extensions to the formalism. We have thus explored
the addition of the empty language $\emptyset$, Kleene star, and a
lexer. The augmented syntax can be seen in Figure
\ref{gen-contig-types}. The addition of $\emptyset$ (via \konst{Void})
and Kleene star (via \konst{List}) has been accomplished, along with
the proofs verifying the upgraded matcher. We also discuss replacing
the existing base types with a lexer for completeness, even though that
discussion more appropriately belongs to future work.

\begin{figure}
\[
\begin{array}{rcl}
 \tau & =    & \konst{Base}\; (\mathit{regexp} \times \mathit{valFn}) \\
      & \mid & \konst{Void} \\
      & \mid & \konst{List}\; \tau \\
      & \mid & \konst{Recd}\; (f_1 : \tau_1) \ldots (f_n : \tau_n) \\
      & \mid & \konst{Array}\; \tau \; \mathit{exp} \\
      & \mid & \konst{Alt}\; \mathit{bexp} \; \tau_1\; \tau_2
\end{array}
\]
\caption{Extended contiguity types}
\label{gen-contig-types}
\end{figure}

\begin{definition}[Semantics additions]

A base type element has the form
$\konst{Base}(\mathit{regexp},\mathit{valFn})$, where $\mathit{valFn}$
is a function that maps the string recognized by $\mathit{regexp}$ to
a data value.  The semantics of a \konst{Base} type is just the
(formal language) semantics of its regular expression, \konst{Void}
denotes the empty set, and \konst{List} is a ``tagged'' version of
Kleene star (\konst{NilTag} and \konst{ConsTag} are described in
Section \ref{list-type}).

\[
\LangTheta{\tau} =
\mathtt{case}\; \tau\
 \left\{
 \begin{array}{l}
 \konst{Base}\; (\mathit{regexp}, \mathit{valFn}) \Rightarrow  \Lang{\mathit{regexp}} \\

 \konst{Void} \Rightarrow  \emptyset \\

 \konst{List}\; \tau \Rightarrow
   (\konst{ConsTag}\cdot \LangTheta{\tau})^{*} \cdot \konst{NilTag} \\

 ... \text{(previous clauses)}  \\

 \end{array}
 \right.
\]
\end{definition}

\subsection{\konst{Void} and in-message assertions}\label{assert}

The \konst{Alt} constructor combined with \konst{Void} supports an
\emph{in-message assertion} feature. The contiguity type
$\konst{Assert}\;\mathit{bexp}$ is defined as follows:

\begin{definition}[\konst{Assert}]

\[
  \konst{Assert}\; \mathit{bexp} =
     \konst{Alt} \; \mathit{bexp} \; (\konst{Recd}\, [\,]) \; \konst{Void}
\]

\end{definition}

\noindent The meaning of $\konst{Assert}\;b$ is obtained by simplification:
%
\[
\LangTheta{\konst{Assert}\; \mathit{bexp}} =
 \itelse{\konst{evalBexp}\;\theta\;\mathit{bexp}}
        {\varepsilon} {\emptyset}
\]
%
and \konst{match} evaluates it by failing when
$\konst{evalBexp}\;\theta\;\mathit{bexp}$ is false, and otherwise
continuing on without advancing in the input.

\begin{example}[$A^n B^n C^n$]
It is well known that $L = A^n B^n C^n$ is not a context-free
language. We can use \konst{Assert} to specify a language that is
\emph{nearly} $L$ with the following contiguity type:
%
{\small
\begin{verbatim}
  charA = {ch : char, isA : Assert (ch = 65)}   (* "A" = ASCII 65 *)
  charB = {ch : char, isB : Assert (ch = 66)}
  charC = {ch : char, isC : Assert (ch = 67)}

  mesg = {len : u16
          A : charA [len]
          B : charB [len]
          C : charC [len]
        }
\end{verbatim}
}
\noindent In fact $\LangTheta{\konst{mesg}} = u \cdot A^{\konst{toN}(u)} \cdot
B^{\konst{toN}(u)} \cdot C^{\konst{toN}(u)}$.
\end{example}

%% Mirroring the semantics, almost no change to the matcher function is
%% needed to support \konst{Assert} forms: the matcher's traversal of
%% \konst{contig} and string arguments is unchanged, and the accumulation of
%% contextual information in the form of the map $\theta$ also is
%% unchanged. Each \konst{Assert} expression is evaluated when it is
%% encountered; if the result is true, processing continues, otherwise
%% an error has been found.

%% Notice that successful match of a message against a \konst{contig} amounts to
%% a minimal well-formedness check: the message string is long enough to
%% match the \konst{contig}. \konst{Assert} fields provide a basis for deeper
%% semantic checks. It is also worth noting that certain features of
%% \konst{contig}s require checks no matter what kind of user-specified
%% properties are required: both enumerations and arrays have
%% side-conditions that \konst{Assert} can enforce.

\noindent \konst{Assert} expressions have been used extensively when
specifying wellformedness properties for messages in our application
work. The applications include restrictions on array sizes and
constraints on array elements, \eg, requiring that every element in an
array of GPS coordinates is an acceptable GPS coordinate.

\begin{example}[Array limits]

In UxAS messages (see Section \ref{application}) the length of every
array element is held in a separate \emph{length} field which is two
bytes in size. Thus the following \konst{contig}, in the absence of
any further constraint, supports arrays of length up to 65536
elements. A receiver system may well not be prepared for messages
having collections of such potentially large components.
{\small
\begin{verbatim}
  { len : u16
    elts : i32 [len] }
\end{verbatim}
}
\noindent In the meta-data for such messages, one can sometimes find
information about the maximum expected size, usually a fairly small
number. This can be directly expressed inside the \konst{contig} with
an \konst{Assert}:
{\small
\begin{verbatim}
  { len : u16
    len-range : Assert (len <= 8)
    elts : i32 [len] }
\end{verbatim}
}
\noindent Note that the expected array length should be specified
before the array itself, otherwise the allocation attempt might be
made before the check.
\end{example}

\subsection{Kleene Star}\label{list-type}

Although the use of bounded \konst{Array} types provides much
expressiveness for representing sequences of data, ultimately some
kinds of message can not be handled, \ie, those where there is no way to
predict the number of nestings of structure: s-expressions, logical
formulae, and programming language syntax trees are some typical
examples. We address this shortcoming by adding a new contiguity type
constructor---\konst{List}--- of unbounded lists. A message matching a
$\konst{List}\;\tau$ type will be subject to an encoding similar to
implementations of lists in functional languages. The matching
algorithm for contiguity types is extended to handle \konst{List}
objects by iteratively unrolling the recursive equation
%
\[ \kstar{L} = \varepsilon \cup L \cdot \kstar{L} \]
%
Indeed the type $\konst{List}\;\tau$ is represented by the following
contiguity type, a recursive record:

\[
 \konst{List}\;\tau =
   \left\{
     \begin{array}{lcl}
       \konst{tag} & : & \konst{u8} \\
       \konst{test} & : &
       \begin{array}[t] {lcl}
         \konst{Alt} \quad (\konst{tag} = \konst{NilTag}) & \longrightarrow & \varepsilon \\
\phantom{\konst{Alt}} \quad (\konst{tag} = \konst{ConsTag}) & \longrightarrow &
          \{ \konst{hd} : \tau, \ \konst{tl} : \konst{List}\; \tau \} \\
\phantom{\konst{Alt}} \quad \text{otherwise} & \longrightarrow & \konst{Void}
        \end{array}
     \end{array}
   \right.
\]

In words, a $\konst{List}\;\tau$ matches a sequence of records where
a single-byte tag (\konst{NilTag} or \konst{ConsTag}) is read, then tested to see
whether to stop parsing the list (\konst{NilTag}) or to continue on to
parse a $\tau$ into the \konst{hd} field and recurse in order to
process the remainder of the list. An incorrect value for the tag results in failure.
Thus, the list of integers
%
\[ \konst{Cons}(1, \konst{Cons}(2, \konst{Cons}(3, \konst{Nil}))) \]
%
can be represented in a message as (assume \konst{Code} is an encoder for integers)
%
\[ \konst{ConsTag}\cdot \konst{Code}(1) \cdot
   \konst{ConsTag}\cdot \konst{Code}(2) \cdot
   \konst{ConsTag}\cdot \konst{Code}(3) \cdot \konst{NilTag}
\]
%
and \konst{match}, given type $\konst{List}\; \konst{int}$
(where $\konst{int}$ is a contiguity type for some flavor of integer)
succeeds, returning the context
%
\[
\begin{array}{rcl}
\konst{root.tag} & \mapsto & \konst{ConsTag} \\
\konst{root.hd} & \mapsto & \konst{Code}(1) \\
\konst{root.tl.tag} & \mapsto & \konst{ConsTag} \\
\konst{root.tl.hd} & \mapsto & \konst{Code}(2) \\
\konst{root.tl.tl.tag} & \mapsto & \konst{ConsTag} \\
\konst{root.tl.tl.hd} & \mapsto & \konst{Code}(3) \\
\konst{root.tl.tl.tl.tag} & \mapsto & \konst{NilTag}
\end{array}
\]

This solution is compositional, in the sense that \konst{List} types
can be the arguments of other contiguity types, can be applied to
themselves, \eg, $\konst{List}(\konst{List}\;\tau)$, \etc \; Thus
quite general branching structures of arbitrary finite depth and width can
be specified and parsed with this extension.  The approach captures a
certain class of \emph{context-free-like} languages. However, it
differs distinctly from the standard Chomsky hierarchy, mainly because
sums are determined by \emph{looking behind} when computing which
choice to follow in an \konst{Alt} type; for example, the list parser
branches \emph{after} it has seen the tag. The similarity with
`no-lookahead' parsing, such as LL(0) and LR(0), deserves further
investigation.

\begin{example}[First order term challenge]

Although lists of contig types can be straightforwardly constructed
with the above encoding, there remains a problem when lists are part
of a recursive construction. A classic example is \emph{first order
  terms}, as described by the following ML-style datatype:

{\small
\begin{verbatim}
  term = Var of string
       | App of string * term list
\end{verbatim}}

\noindent The following contiguity types capture a binary encoding of
\verb+term+, using tags to distinguish the two kinds of term:
\[
\begin{array}{l}
  \konst{string} = \{\konst{len} : \konst{u16}, \konst{elts} : \konst{char}\, [\konst{len}] \}
\\ \\
 \konst{term} =
   \left\{
     \begin{array}{lcl}
       \konst{tag} & : & \konst{u8} \\
       \konst{test} & : &
       \begin{array}[t] {lcl}
         \konst{Alt} \quad (\konst{tag} = \konst{VarTag})
                      & \longrightarrow
                      & \{\konst{varName} : \konst{string}\} \\
\phantom{\konst{Alt}} \quad (\konst{tag} = \konst{AppTag})
                  & \longrightarrow &
          \{ \konst{fnName} : \konst{string}, \ \konst{Args} : \konst{List}\; \konst{term} \} \\
\phantom{\konst{Alt}} \quad \text{otherwise} & \longrightarrow & \konst{Void}
        \end{array}
     \end{array}
   \right.
\end{array}
\]

However, such \emph{nested recursive} specifications demand more
elaborate constructions, for example treating \konst{term} and
\konst{List} as being mutually recursively defined, as is already done
for nested recursive datatypes in theorem provers \cite{jrh:ind-defs}.
\end{example}

\subsection{Lexing}

Currently, the set of base contiguity types comprises the usual base
types expected in most programming languages. Semantically, a base
type denotes a set of strings of the specified width, but it is also
coupled with an \emph{interpretation function} for example, the
contiguity type \verb+u8+ denotes the set of all one-byte strings,
interpreted by the usual unsigned valuation function:
%
\[ \konst{u8} = (\set{s \mid \konst{length}(s) = 1}, \konst{toN} )
\]
%
This approach cannot, however, capture base types such as string
literals of arbitrary size, or bignums, or the situation in packed
bit-level encodings where fields are of \emph{ad hoc} sizes aimed at
saving space. A common generalization is to express base types via
regular expressions paired with interpretation functions. In that
setting \konst{u8} can be defined as
%
\[ \konst{u8} = ( . \; , \konst{toN})
\]
%
(where `.' is the standard regular expression denoting any character). Similarly,
%
\[ \konst{Cstring} = ([\backslash 001-\backslash 255]^{*}\backslash 000, \lambda x.\,x )
\]
%
denotes the set of zero-terminated strings, as found in the C
language. Its displayed interpretation is just the identity function,
but could just as well be a function that drops the terminating
\verb+\000+ character.

Scott Owens has already formalized a HOL4 theory of regexp based,
maximal munch, lexer generation, and it is future work to adapt
contiguity types to use those lexemes instead of the current
restricted set of base types.

\section{Application}\label{application}

In the DARPA CASE project, we have been applying contiguity types to
help create provably secure message filters and runtime monitors. One
example we have been working with is \emph{OpenUxAS}, which has been
developed by the Air Force Research Laboratory.\footnote{See
  \texttt{https://github.com/afrl-rq/OpenUxAS}.}  UxAS is a collection
of modular services that interact via a common message-passing
architecture, aimed at unmanned autonomous systems.  Each service
subscribes to messages in the system and responds to queries. The
content of each message conforms to the Light-weight Message Control
Protocol (LMCP) format. In UxAS, software classes providing LMCP
message creation, access, and serialization/deserialization are
automatically generated from XML descriptions, which detail the exact
data fields, units, and default values for each message. All UxAS
services communicate with LMCP formatted messages.

An example LMCP message type is \konst{AirVehicleState}. The following
is its contiguity type:

{\small
\begin{verbatim}
AirVehicleState =
   {EntityState   : EntityState
    Airspeed      : float
    VerticalSpeed : float
    WindSpeed     : float
    WindDirection : float
   }
\end{verbatim}
}
\noindent where an \konst{EntityState} is quite an elaborate type:
{\small
\begin{verbatim}
   { ID : i64
     u  : float
     v  : float
     w  : float
     udot : float
     vdot : float
     wdot : float
     Heading : float
     Pitch   : float
     Roll    : float
     p : float,  q : float, r : float
     Course : float
     Groundspeed : float
     Location : mesgOption "LOCATION3D" location3D
     EnergyAvailable : float
     ActualEnergyRate : float
     PayloadStateList
        : uxasBoundedArray (mesgOption "PAYLOADSTATE" payloadState) 8
     CurrentWaypoint : i64
     CurrentCommand  : i64
     Mode : uxasNavigationMode
     AssociatedTasks : uxasBoundedArray i64 8
     Time : i64
     Info : uxasBoundedArray(mesgOption "KEYVALUEPAIR" keyValuePair) 32
   }
\end{verbatim}
}

\noindent Most of the fields are simple base types, but the \konst{Location}
field and the \konst{PayloadStateList} are complex. They are expressed
with some derived syntax, which we will explain.

\begin{itemize}
\item The \konst{Location} field,
  {\small\verb+mesgOption "LOCATION3D" location3D+}, may or may not
  occur (signalled with a tag field), but if it does, it is a
  \konst{location3D}, which is a GPS location, and its latitude,
  longitude, and altitude fields are checked with an \konst{Assert} to
  make sure they lie within the expected numeric ranges, which are
  expressed as floating point numbers.

{\small
\begin{verbatim}
  AltitudeType = AGL | MSL

  location3D = {
   Latitude  : double,
   Longitude : double,
   Altitude  : float,
   AltitudeType : AltitudeType,
   Wellformed : Assert (
       -90.0 <= Latitude <= 90.0 and
      -180.0 <= Longitude <= 180.0 and
       0.0 <= Altitude <= 15000.0)
  }
\end{verbatim}}

\item The \konst{PayloadStateList} is a variable-length array of
  optional records, with maximum length 8. Each record has, along with
  other fields, its own variable-length array of key-value pairs, and
  the key and value of each such pair is a variable-length string.

\end{itemize}

We have formalized most of the LMCP messages as contiguity types, and
created filters and parsers by instantiating the \konst{match}
algorithm. In order to meet the demands of LMCP message modelling, the
matcher algorithm has been upgraded to support a fuller expression and
boolean expression language, but the core algorithm is the same as our
verified core version. The filters and parsers have been added to an
existing UxAS design and successfully tested with the UxAS simulator.

\section{Extensions and future work}

Various extensions have been easy to add to the contiguity type
framework, and we also have more substantial ideas to pursue for
future work.

\begin{description}

\item [Enumerations] An \emph{enumeration} declaration
introduces a new base contiguity type, and also adds the specified
elements to a map associating constant names to numbers.
Suppose that enumerations are allowed to have up to 256 elements,
allowing any enumerated element to fit in one byte. The following
enumeration is taken from UxAS messages:
{\small
\begin{verbatim}
  NavigationMode
    = Waypoint | Loiter | FlightDirector
    | TargetTrack | FollowLeader | LostComm
\end{verbatim}
}
A field expecting a \konst{NavigationMode} element will be one byte wide,
and thus there are 250 byte patterns that should not be allowed in the
field. Thus, the \konst{contig}

{\small\begin{verbatim}
  { A : NavigationMode }
\end{verbatim}
}
should be replaced by

{\small
\begin{verbatim}
  { A : NavigationMode
    A-range : Assert (A <= 5)
  }
\end{verbatim}
}

\item [Raw blocks] A raw chunk of a string (byte array) of a size that
  can depend on the values of earlier fields is easy to specify:
  \[ \konst{Raw}\; \mathit{exp} \]

For example, a large \konst{Array} form can lead to a large number of
L-values being stored in $\theta$; if none are ever accessed later,
\eg, if the array is some image data, it can be preferable to simply
declare a \konst{Raw} block. Thus a 2D array can be blocked out in the
following manner:

{\small
\begin{verbatim}
  { rows : i32
    cols : i32
    block : Raw (rows * cols)
  }
\end{verbatim}
}

\item [Guest scanners] It seemed useful to provide a general ability
  to host scanning functions. This is accomplished via the following
  constructor:

  \[ \konst{Scanner}\;
     (\mathit{scanfn} : \konst{string} \to (\konst{string} \times \konst{string})\konst{option}) \]

  When a custom scanner is encountered during the matching process,
  the scanner is invoked on the input and should either fail or
  provide an $(s_1,s_2)$ pair representing a splitting of the
  input. Then $s_1$ is added to $\theta$ at the current \lval, and
  matching continues on $s_2$.

\item [Non-copying implementations] In the discussion so far, we have
  assumed that the input string is being broken up into substrings
  that are placed into the \lval{} map $\theta$. However, very little
  is changed if, instead of a substring, an \lval{} in $\theta$ maps
  to a pair of indices $(\mathit{pos},\mathit{width})$ designating the
  location of the substring. The result is a matcher that never copies
  byte buffer data. This is necessary to synthesize efficient filters.

  In making this representation change, there is a slight change to
  the semantics. In the original, $\theta(\lval)$ yields a string
  whereas in the non-copying version, $\theta(\lval)$ yields a pair of
  indices, which means that the original string $\mathit{str}_0$ needs
  to be included in applying the assignment.

\item [Compilation] Our current implementation of contiguity types is
  in an \emph{interpreted} style: the evaluation of numeric
  \konst{Array} bounds and boolean guards on \konst{Alt} conditions is
  done with respect to the current context, which is explicitly
  accumulated as the message is processed. However, notice that the
  host programming language will no doubt already provide compilation
  for numeric and boolean expressions. This leads to the idea of
  \emph{compiling contiguity types}: the current matching algorithm
  for contiguity types can be replaced by the generation of equivalent
  host-language code which is then compiled by the host-language
  compiler and evaluated. Although the current contiguity type matcher
  has proved to be fast enough to keep up with the real-time demands
  of the UxAS system, we expect that a compiled version of the code
  would be much faster.

 We would like to formalize the compilation algorithm and prove it
 correct. Since the CakeML\cite{cakeml:popl14} formalization provides
 an operational semantics for CakeML programs, and a convenient
 translation from HOL4 expressions to CakeML ASTs, one should be able
 to prove a correctness theorem relating the matcher function of the
 present paper with a compiler that takes a contiguity type and
 generates CakeML. However, this is only speculative; there are many
 details to work out.

\item [Relationship with grammars]

Contiguity types and \konst{match} provide a type-directed and
context-oriented parser generator that has some similarities with
LL(0) or LR(0) languages wherein the parser can proceed with no
lookahead. This is useful for binary-encoded datastructures. It would
be very interesting to attempt to bridge the gap with conventional
parsing technology based on grammars. A good beginning would be to
understand the issues involved in attempting to translate context-free
grammars into contiguity types and \emph{vice versa}.

\end{description}

\section {Related work}

As mentioned in the introduction, domain specific languages for
message formats have been around for a long time. Semantic definitions
and verification for them is a much more recent phenomenon. The PADS
framework \cite{pads:ddl} aimed at supporting a wide variety of
formats, including text-based. Its core message description formalism
was given semantics by translation into dependent type theory. An
interesting integration of context-sensitivity into a conventional
grammar framework has been done by Jim and Mandelbaum
\cite{trevor-jim-dependent-parsing}. Everparse \cite{everparse} is an
impressive approach, based on parser combinators and having an
emphasis on proving the invertibility of encode/decode pairs with
automated proof (other properties are also established). Chlipala and
colleagues \cite{narcissus-encode-decode} similarly emphasize
encode/decode proofs, basing their work in Coq and using the power of
dependent types to good effect. Formats based on dependent types can
(and do) use the built-in expressive power of type theory to enforce
semantic properties on data. A recent language in this vein is Parsely
\cite{parsley:sri} which leverages the dependent records and predicate
subtyping of PVS to provide a combination of PEG parsing and attribute
grammars aimed at parsing complex language formats.

Many of these efforts obtain the semantics of the data description
formalism by translation into features provided by a powerful host
logic. In contrast, contiguity types use only very basic---and easily
implemented---concepts. This means that contiguity type matchers,
parsers, and extensions can be directly implemented in any convenient
programming language.  Contiguity types also provide a kind of
dependency, without leveraging the type system of the theorem
prover. We suspect that working at the representation level, and using
L-values, allows one to get some of the benefits of type
dependency. Another distinguishing aspect of our work is that our
emphasis on filters means that we are primarily interested in the
enforcement of semantic properties on message contents rather than
encode/decode properties. In future work we expect to be able to
leverage this in high performance filter implementations.


%Other papers toreference \cite{verified:protocolbuffer}.

\section{Conclusion}

 We have designed, formalized, proved correct, implemented, and applied
a specification language for message formats, based on formal
languages and the venerable notion of L- and R-values from imperative
programming. The notion of contiguity type seems to give a lot of
expressive power, sufficient to tackle difficult idioms in
self-describing formats. Contiguity types integrate common structuring
mechanisms from programming languages, such as arrays, records, and
lists while keeping the foundation in sets of strings, which seems
appropriate for message specifications.

%% Bibliography
\bibliography{contig}

\end{document}
