%\documentclass[svgnames]{llncs}
\documentclass[sigplan,10pt,anonymous,review]{acmart}\settopmatter{printfolios=true,printccs=false,printacmref=false}
\newcommand{\ie}{\textit{i.e.}}
\newcommand{\eg}{\textit{e.g.}}
\newcommand{\etal}{\textit{et al.}}
\newcommand{\etc}{\textit{etc.}}
\newcommand{\adhoc}{\textit{ad hoc}}

\usepackage{xspace}
%\usepackage{microtype}

% Packages and abbreviations used by Konrad
\usepackage{latexsym}
\usepackage{amsmath}
\usepackage{amssymb}
\usepackage{amsbsy}

\newcommand{\konst}[1]{\ensuremath{\mathsf{#1}}}
\newcommand{\imp}{\Rightarrow}
\newcommand{\lval}{\ensuremath{\mathit{lval}}}
\newcommand{\set}[1]{\ensuremath{\{ {#1} \}}}
\newcommand{\kstar}[1]{\ensuremath{{#1}^{*}}}
\newcommand{\Lang}[1]{\ensuremath{{\mathcal L}({#1})}}
\newcommand{\LangTheta}[1]{\ensuremath{{\mathcal L}_{\theta}({#1})}}
\newcommand{\itelse}[3]{\mbox{$\mathtt{if}\ {#1}\ \mathtt{then}\ {#2}\ \mathtt{else}\ {#3}$}}

% Latex trickery for infix div operator, from stackexchange

\makeatletter
\newcommand*{\bdiv}{%
  \nonscript\mskip-\medmuskip\mkern5mu%
  \mathbin{\operator@font div}\penalty900\mkern5mu%
  \nonscript\mskip-\medmuskip
}
\makeatother

% For the diagrams, pinched from the Data61 report

\usepackage{graphicx}
\usepackage{tikz}
\usetikzlibrary{positioning}
\usetikzlibrary{arrows}
\usetikzlibrary{shapes.multipart}

\begin{document}


%% Title information
\title{Specifying Message Formats with \\ Contiguity Types}

\author{Konrad Slind}
\affiliation{Trusted Systems Group Collins Aerospace}

\begin{abstract}
We introduce \emph{Contiguity Types}, a formalism for network message
formats. Contiguity types provide an intermediate layer between
programming language data structures and messages, offering a helpful
setting from which to automatically generate decoders, filters, and
message generators. L-values and R-values provide valuable
support in handling common self-describing message formats.
\end{abstract}

\maketitle


\section{Introduction}\label{sec:intro}

Serialized data, for example network messages, is an important
component in many computer systems.  As a result, innumerable
libraries and tools have been created that use high level
specifications as a basis for automating the creation, validation, and
decoding of such data.\footnote{We will use \emph{message} to stand in
  for any such format.} Usually, these high level specifications
describe the format of a message in terms of how the elements (fields)
of the message are packed side-by-side to make the full message. When
the size of each field is known in advance, there are really no
conceptual difficulties. However, messages can be more complicated
than that.

The main source of difficulty is \emph{self-describing} messages:
those where information embedded in fields of the message determines
the final structure of the message. Two of the main culprits are
variable-length arrays and unions. A \emph{variable-length array} is a
field where the number of elements in the field depends on the value
of some already-seen field (or, more generally, as the result of a
computation involving previously-seen information in the message).
The length is therefore a value determined at runtime. A \emph{union}
is deployed when some information held in a message is used to
determine the structure of later portions of the message. For example,
unions can be used to support versioning where version $i$ has $n$
fields, and version $i+1$ has $n+1$. In settings where both versions
need to be supported in a single format, it can make sense to encode
the version handling inside the message, and unions are how this can
be specified.

We want to believe that tools and techniques from formal language
theory such as regular expressions, automata, grammars, parser
generators, \etc\, can provide an effective way to tackle message
formats. \footnote{And have been using the acronym \konst{SPLAT}
  (Semantic Properties for Language and Automata Theory) to refer to
  this approach.} For example, we have used regular expressions as a
specification language for message formats having simple numeric
interval constraints on the values allowed in fields. Generation of
the corresponding DFA results in an efficient table-driven automaton
implementing the specified constraints \cite{}, with a formal proof
certificate connecting the original constraints with the DFA behavior.

However, self-describing data formats fall outside the realm of common
formal language techniques; \eg, variable-length fields clearly aren't
able to be described by regular or context-free languages. Why? These
language classes encompass repetitions of a fixed or unbounded size,
but not repetitions of a size determined by parts of the input
string. It does seem that context-sensitive grammars can, in
principle, specify such information, but there are few tools
supporting context sensitive languages. Another possibility would be
to use \emph{parser combinators} in order to quickly stitch together a
parser; it is likely that the combinators can be instrumented to
gather and propagate contextual information. However, we are seeking a
high level of formal specification and automation, while still being
rooted in formal languages, with their emphasis on sets of strings as
the basic notion.

\section {Contiguity Types}

The characteristic property of a message is \emph{contiguity}: all the
elements of the message are laid out side-by-side in a byte array (or
string). Our intuition is that a message is the \emph{result} of
encoding structured data and we want to represent a version of that
original structure while not being tied to any particular programming
language. Figure \ref{contig-types} defines the syntax of contiguity
types. Starting with common base types (booleans, characters, signed
and unsigned integers, \etc), compound structures are built with
records, arrays, and unions. (We will use the terms $\tau$,
\konst{contig}, and \emph{contiguity type} interchangeably.)

\begin{figure}
\label{contig-types}
\[
\begin{array}{rcl}
 \mathit{base} & = & \konst{bool} \mid \konst{char} \mid \konst{u8} \mid
 \konst{u16} \mid \konst{u32} \mid \konst{u64}  \ \mid \\
 & & \konst{i16} \mid \konst{i32} \mid \konst{i64} \mid \konst{float} \mid \konst{double} \\
 \tau & = & \mathit{base} \\
      & \mid & \konst{Recd}\; (f_1 : \tau_1) \ldots (f_n : \tau_n) \\
      & \mid & \konst{Array}\; \tau \; \mathit{exp} \\
      & \mid & \konst{Union}\; (\mathit{bexp}_1 : \tau_1) \ldots (\mathit{bexp}_n : \tau_n)
\end{array}
\]
\caption{Syntax of contiguity types}
\end{figure}

Notice that $\tau$ is defined in terms of a type of arithmetic
expressions $\mathit{exp}$ and also $\mathit{bexp}$, boolean
expressions built from $\mathit{exp}$. Now consider
\[
 \konst{Array} \; \tau \; \mathit{exp} \ .
\]
For this to specify a varying length array dependent on
other fields of the message, its dimension $\mathit{exp}$ should be
able to refer to the \emph{values} of those fields. The challenge is
just how to express the concept of ``other fields'', \ie, we need a
notation to describe the \emph{location} in the message buffer where
the value of a field can be accessed. Our core insight is that this is
similar to a problem that programming language designers had in the
60s and 70s, resolved by the notions of \emph{L-value} and
\emph{R-value}. The idea is originally due to Christopher Strachey in
CPL and developed subsequently, for example by Dennis Ritchie in
C.\footnote{https://www.bell-labs.com/usr/dmr/www/chist.html}

Before getting into formal details, we discuss a few examples.  We
will use familiar notation: records are lists of $\mathit{name} :
\tau$ elements enclosed by braces while an array field
$\konst{Array}\;c\;\mathit{dim}$ is written
$c [\mathit{dim}]$.\footnote{Our current toolset for \konst{contig}s supports
the syntax on output but we have yet to implement a parser.}

\begin{enumerate}

\item The following is a record with no self-describing aspects: each
  field is of a statically known size.

\begin{verbatim}
  {A : u8
   B : {name : char [13]
        cell : i32}
   C : bool
  }
\end{verbatim}

The \verb+A+ field is specified to be an unsigned int of width 8 bits,
the \verb+B+ field is a record, the first element of which is a
character array of size 13, and the second element of which is a 32
bit integer; the last field is specified to be a
boolean.

\item Variable-sized strings are a classic self-describing aspect. In
  this example the contents of the \verb+len+ field determines the
  number of elements in the \verb+elts+ field.

\begin{verbatim}
  { len : u16
    elts : char [len]
  }
\end{verbatim}


\item The following examples shows the \konst{Union} construct being
  used to support multiple versions in a single format.  Messages with
  the value of field \verb+versionID+ being less than 14 have three
  fields in the message, while all others have two.

\begin{verbatim}
  {versionID : u8
   versions : Union {
      (versionID < 14,  { A : i32, B : u16})
      (versionID >= 14, { Vec : char [13]})
    }
  }
\end{verbatim}

\item The following is a contrived example showing the need for
  resolution of multiple similarly named fields; it also shows how the
  information needed to determine the message structure may be deeply
  buried in some fields.

\begin{verbatim}
  {len : u16
   A : {len : u16
        elts : u16[len]
       }
   B : char [A.len * len]
   C : i32 [A.elts[0]]
  }
\end{verbatim}

\end{enumerate}

\subsection{Expressions, L-values, and R-values}

In programming languages, an \emph{L-value} is an expression that can
occur on the left-hand side of an assignment statement. Similarly,
\emph{R-value} designates expressions occurring on the right-hand side
of assignments. Following are a few examples:

\begin{verbatim}
    x := x + 1
    A[x] := B.y + 42
    A[x].lens.fst[7] := MAX_LEN * 1024 + B.y
\end{verbatim}

Figure \ref{Lvalues} presents the formal syntax for L-values, R-values,
and the boolean expressions we will use.  An L-value can be
a variable, an array index, or a record field access. R-values are
arithmetic expressions that can contain L-values (we will use
$\mathit{exp}$ interchangeably with R-value).

\begin{figure}
\[
\begin{array}{rcl}
\mathit{lval} & = & \mathit{varname} \mid
                    \mathit{lval} \, [ \mathit{exp} ] \mid
                    \mathit{lval} . \mathit{fieldname} \\
  & & \\
\mathit{exp} & = & \konst{Loc}\; \mathit{lval}
              \mid \konst{nLit}\; \konst{nat}
              \mid \mathit{constname}
              \mid \mathit{exp} + \mathit{exp}
              \mid \mathit{exp} * \mathit{exp} \\
  & & \\
\mathit{bexp} & = & \konst{bLit}\; bool
              \mid  \neg \mathit{bexp}
              \mid  \mathit{bexp} \lor \mathit{bexp}
              \mid  \mathit{bexp} \land \mathit{bexp}
              \mid  \mathit{exp} = \mathit{exp}
              \mid  \mathit{exp} < \mathit{exp}
\end{array}
\]
\caption{L-values, expressions, and boolean expressions}
\label{Lvalues}
\end{figure}

An L-value denotes an \emph{offset} from the beginning of a
data structure, plus a \emph{width}. In an R-value, an occurrence of an
L-value is mapped to the value of the patch of memory between
$\mathit{offset}$ and $\mathit{offset} + \mathit{width}$. For the
purpose of specifying message formats, it may not be immediately
obvious that a notation supporting assignment in imperative languages
can help, but there is indeed a form of assignment lurking.

The above explanation of L-values centers on indices into a byte
buffer; in the following we will give a mild variant of this: instead
of indices into the buffer, we lift out the designated slices. Thus, given
environments $\theta: \mathit{lval} \to \konst{string}$ (binding
L-values to strings), $\Delta : \konst{string} \to \mathbb{N}$
(binding constant names to numbers) and function
$\konst{toN}:\konst{string}\to\mathbb{N}$ (which interprets byte
sequences to numbers), expression evaluation and boolean expression
evaluation have conventional definitions:

\[
\begin{array}{l}
\konst{evalExp} \; e =
\mathtt{case}\; e\
 \left\{
 \begin{array}{lcl}
    \konst{Loc}\; \lval & \Rightarrow & \konst{toN}(\theta(\lval)) \\
    \konst{nLit}\; n & \Rightarrow & n  \\
    \mathit{constname} & \Rightarrow & \Delta(\mathit{constname})  \\
    e_1 + e_2 & \Rightarrow & \konst{evalExp}\; e_1 + \konst{evalExp}\; e_2  \\
    e_1 * e_2 & \Rightarrow & \konst{evalExp}\; e_1 * \konst{evalExp}\; e_2  \\
  \end{array}
 \right.
 \\ \\
\konst{evalBexp} \; b =
\mathtt{case}\; b\
 \left\{
 \begin{array}{lcl}
    \konst{bLit}\; b_1 & \Rightarrow & b_1 \\
    \neg b_1 & \Rightarrow & \neg(\konst{evalBexp} \; b_1)  \\
    b_1 \lor b_2 & \Rightarrow & \konst{evalBexp} \;b_1 \lor \konst{evalBexp} \;b_2   \\
    b_1 \land b_2 & \Rightarrow & \konst{evalBexp} \;b_1 \land \konst{evalBexp} \;b_2   \\
    e_1 = e_2 & \Rightarrow & \konst{evalExp} \;e_1 = \konst{evalExp} \;e_2   \\
    e_1 < e_2 & \Rightarrow & \konst{evalExp} \;e_1 < \konst{evalExp} \;e_2
  \end{array}
 \right.

\end{array}
\]


\subsection{Semantics}

 We now confess to misleading the reader: in spite of the notational
 similarity, a contiguity type is \emph{not} a type: it is a formal
 language. A type is usually understood to represent a set (or domain)
 of values, \eg, the type \konst{int32} represents a set of
 integers. In contrast, the contiguity type \konst{i32} represents the
 set of strings of width 32 bits. An element of a contiguity type can
 be turned into an element of a type by providing interpretations for
 all the strings at the leaves and interpreting the \konst{Recd} and
 \konst{Array} constructors into the corresponding type constructs. (A
 base contiguity type therefore serves mainly as a \emph{tag} to be
 interpreted as a width and also as an intended target type.) Thus,
 contiguity types sit---conveniently---between the types in a
 programming language and the strings used to make messages.

The semantics definition depends on a few basic notions familiar from
language theory: language concatenation, and iterated language
concatenation.

\begin{align*}
L_1 \cdot L_2 &= \set{w_1 w_2 \mid w_1 \in L_1 \land w_2 \in  L_2} \\
L^0 &= \varepsilon \\
L^{n+1} &= L \cdot L^n
\end{align*}


\begin{definition}[Semantics of contiguity types]

 In the following, we assume given an assignment $\theta$ adequate for
evaluating expressions, \ie, having bindings for every
$\mathit{lval}$ in every $\mathit{exp}$ and $\mathit{bexp}$
occurring in $\tau$.

\[
% \begin{array}{l}
\LangTheta{\tau} =
\mathtt{case}\; \tau\
% \hspace*{3mm}
 \left\{
 \begin{array}{l}
 \mathit{base} \Rightarrow \set{s \mid \konst{len}(s) = \konst{width}(base)} \\
 \konst{Recd}\; (f_1 : \tau_1) \ldots (f_n : \tau_n)
      \Rightarrow \LangTheta{\tau_1} \cdot \ldots \cdot \LangTheta{\tau_n}
\\
 \konst{Array}\; \tau_1 \; \mathit{exp}
      \Rightarrow  \LangTheta{\tau_1}^{\konst{evalExp}\;\theta\;\mathit{exp}}
\\
 \konst{Union}\; (\mathit{bexp}_1 : \tau_1) \ldots (\mathit{bexp}_n : \tau_n) \Rightarrow \\
  \hspace*{5mm}
 \left\{
 \begin{array}{ll}
    \LangTheta{\tau_i} &  \mathrm{if}\ \konst{evalBexp}\;\theta\;\mathit{bexp}_i = \konst{true} \\
                  & \mathrm{and\ no\ other}\ \mathit{bexp}_j\ \mathrm{evaluates\ to}\ \konst{true}  \\
    \emptyset & \mathrm{otherwise}
 \end{array}
 \right.
 \\
\end{array}
 \right.
%\end{array}
\]
\end{definition}

\begin{example}
Consider the following schema for an \emph{option} contiguity
type. The empty record \verb+{}+ associated with boolean expression
$b$ has no fields.

\begin{verbatim}
  Union { (b, {})
          (not(b), c)
   }
\end{verbatim}

\noindent In case $\mathit{b}$ evaluates to \konst{true}, no portion of the
string is consumed; otherwise, $c$ specifies the remainder of the
processing. It is instructive to consider how this type works with
arrays. For example, a string meeting the following \konst{contig} specification

\begin{verbatim}
  (Union { (b, {}), (not(b), i32) }) [3]
\end{verbatim}

\noindent is either zero or twelve bytes in length (assuming that
\verb+i32+ is four bytes wide).

\end{example}

\section{Algorithms}

The following problems involving contiguity types are worth
investigating. At present we have been working on decoding and
filtering.

\begin{description}

\item [Decoding] A decoder breaks a sequence of bytes up and puts the
  pieces into a useful data structure, typically a parse tree. We will
  discuss this in more detail in Section \ref{decoding}.

\item [Filtering] Computes an answer to the question: ``does a
  sequence of bytes meet the specification of a given contiguity
  type''. This is an instance of the language recognition problem:
  given a formal specification of a language, decide if a string is in
  the language. More powerful filters enforce that certain fields of a
  message, when interpreted, meet specific semantic properties.

\item [Serialization] Given a contiguity type, synthesize a function
  that writes a compact binary version of a data structure to a message.

\item [Test generation] Given a contiguity type, generate byte
  sequences that do (or do not) meet its specification and feed the
  sequences to implementations in order to observe their behaviour.

\item [Learning] Given training sets of message buffers that are
  accepted/rejected by an implementation, attempt to discover a
  contiguity type for the entire set of messages.

\end{description}

\subsection{Decoding}
\label{decoding}

Above we mentioned that decoding can result in parse trees; however,
self-describing messages allow a different conceptual framework to be
brought to bear. There is an important distinction between
\emph{parsing}, which discovers structure (parse trees), and
\emph{matching}, which is given structure and calculates assignments
(substitutions).\footnote{Thus the notion of matching discussed here
  is in the tradition of term rewriting \cite{nipkow-trs-book}, the
  main difference being that our substitutions are applied to
  $\mathit{lvar}$s rather than variables.} Giving some types helps
make the difference clear:

\begin{align*}
  \konst{parse} &: \mathit{grammar} \to \konst{string} \to \mathit{parse tree} \\
  \konst{match} &: \mathit{AST} \to \konst{string} \to \mathit{assignments}
\end{align*}

For contiguity types, the central decoding algorithm is a
\emph{matcher}: given a contiguity type $\tau$ and a string $s$, the
matcher will either fail, or succeed with an assignment $\theta :
\mathit{lval} \mapsto \konst{string}$ mapping each L-value in $\tau$
to its corresponding slice of $s$. The assignment $\theta$ can be
post-processed to yield a standard parse tree, but its novelty and
strength is that $\theta$ can be dynamically consulted to access the
values needed to guide the processing of self-describing messages.

\begin{definition}[Matching algorithm]

The matching algorithm operates over a triple $(\mathit{worklist},
\mathit{str}, \theta)$ where $\mathit{worklist}$ is a stack used to
linearize the input contiguity type $\tau$, $\mathit{str}$ represents
the remainder of the input string, and $\theta$ is the assignment
being built up. Each element of the $\mathit{worklist}$ is in fact a
$(\tau,\mathit{lval})$ pair, where $\tau$ is a \konst{contig}, and
$\mathit{lval}$ is the path growing down from the root to $\tau$. The
notation $(\lval \mapsto \mathit{slice}) \bullet \theta$ denotes the
addition of binding $\lval \mapsto \mathit{slice}$ to $\theta$.

We examine the cases in turn:

\begin{enumerate}

\item The worklist is empty; the match has been successful.
%
\begin{align*}
([], \mathit{str}, \theta) &\Rightarrow \konst{SOME}(\mathit{str}, \theta)
\end{align*}


\item The first element of the worklist is a base type. Break the
  prescribed number of bytes off the front of the string, giving
  $\mathit{str} = (\mathit{slice},\mathit{rst})$ and insert the
  binding into $\theta$ before recursing. If the string is shorter
  than the requested number of bytes, fail.
%
\begin{align*}
((\konst{Basic}\;a, \lval)::t, \mathit{str}, \theta)
   &\Rightarrow
  (t,\mathit{rst}, (\lval \mapsto \mathit{slice}) \bullet \theta)
\end{align*}

\item The first element of the worklist is $\mathit{recd} =
  \konst{Recd}\;(f_1 : \tau_1) \ldots (f_n : \tau_n)$. Before
  recursing, the fields are pushed onto the stack, extending the path
  to each field element:
%
\begin{align*}
((\mathit{recd}, \lval)::t, \mathit{str}, \theta)
   &\Rightarrow
  ([(\tau_1,\lval.f_1), \cdots , (\tau_n,\lval.f_n)] @ t,\mathit{str}, \theta)
\end{align*}

\item The first element of the worklist is an array. The dimension
  expression is evaluated to get the width $d$, then $d$ copies are
  pushed onto the stack, where each path is extended with the array
  index.
%
\begin{align*}
((\konst{Array}\; \tau \; \mathit{exp},\lval)::t, \mathit{str}, \theta)
   &\Rightarrow
  ([(\tau,\lval[0]), \cdots , (\tau,\lval[d-1])] @ t,\mathit{str}, \theta)
\end{align*}

\item The first element of the worklist is a union. The boolean guards
  are evaluated; if exactly one of them, $b_i$, evaluates to \konst{true}, the
  corresponding $\tau_i$ is pushed on to the stack. Otherwise, fail.
%
\begin{align*}
((\konst{Union}\; (b_1,\tau_1) \cdots (b_n,\tau_n), \lval)::t, \mathit{str}, \theta)
   &\Rightarrow
  ((\tau_i,\lval)::t,\mathit{str}, \theta)
\end{align*}

\end{enumerate}

\noindent The matcher function, \konst{match} begins with an initial state
%
\[
  \mathit{state}_0 = ([(\mathit{root},\tau)],\mathit{str}_0,\emptyset)
\]

where the initial path is a default \lval{} variable named
$\mathit{root}$, the initial string is $\mathit{str}_0$, and the
initial assignment has no bindings.

\end{definition}

\begin{theorem}[Matcher termination]
Termination is a nice application of the multiset ordering to the
worklist component of the state, as the handling of the \konst{Array}
construct is a version of the Hercules-Hydra problem.
\end{theorem}

The correctness statement for the matcher is similar to those found in
the term rewriting literature.

\begin{theorem}[Matcher correctness]

\[
  \konst{match}\; \mathit{state}_0 = \konst{SOME}(\theta, s)
  \imp \theta(\tau) \cdot s = \mathit{str}_0
\]

\noindent We also want to provide a connection to $\LangTheta{\tau}$,
and this is formalized as

\[
  \mathit{str}_0 = s_1 s_2 \land \konst{match}\; \mathit{state}_0 =
  \konst{SOME}(\theta, s_2) \imp s_1 \in \LangTheta{\tau}
\]

In other words, a successful match provides a $\theta$ adequate for
evaluating expressions, and the matched string is indeed in the
language of $\tau$. A theorem going in the other direction, namely,
that every string in $\LangTheta{\tau}$ will be successfully
matched, is also desirable, but requires some thought as to the
$\theta$ that admit strings into $\LangTheta{\tau}$.

\end{theorem}

\begin{example}
Given the \konst{contig}
\begin{verbatim}
   {A : Bool
    B : Char
    len : u16
    elts : i32 [len]
  }
\end{verbatim}
\noindent and an input string (listed in hex)
\begin{verbatim}
  [0wx1, 0wx67, 0wx0, 0wx5, 0wx0, 0wx0, 0wx0, 0wx19, 0wx0, 0wx0,
   0wx9, 0wx34, 0wx0, 0wx0, 0wx30, 0wx39, 0wx0, 0wx0, 0wxD4,
   0wx31, 0wxFF, 0wxFF, 0wxFE, 0wxB3]
\end{verbatim}
created by encoding: the boolean \verb+true+, the letter \verb+g+, the
number 5 (MSB 2 byte unsigned), and the five MSB 4 byte signed twos complement
integers 25, 2356, 12345, 54321, and -333, the matcher creates the
following assignment of $\mathit{lval}$s to substrings of the input
(each element of the list is of the form
$(\mathit{lval}, (\mathit{tag},\mathit{bytes}))$):

\begin{verbatim}
 [(root.A,       (Bool, [0wx1])),
  (root.B,       (Char, [0wx67])),
  (root.len,     (u16,  [0wx0, 0wx5])),
  (root.elts[0], (i32,  [0wx0, 0wx0, 0wx0, 0wx19])),
  (root.elts[1], (i32,  [0wx0, 0wx0, 0wx9, 0wx34])),
  (root.elts[2], (i32,  [0wx0, 0wx0, 0wx30, 0wx39])),
  (root.elts[3], (i32,  [0wx0, 0wx0, 0wxD4, 0wx31])),
  (root.elts[4], (i32,  [0wxFF, 0wxFF, 0wxFE, 0wxB3]))
 ]
\end{verbatim}

\end{example}

Thus the matcher will break up the input string in accordance with the
specification; the execution, in effect, generates a sequence of
assignments that, if applied, would populate a data structure with the
specified data in the specified places. Therefore it is not really
necessary to generate parse trees to in order to decode messages: one
merely needs a target data structure to write data into. (In fact,
when filtering, no target data structure is needed at all.) The
correctness property will ensure that \emph{all} fields are written
with the specified data. The assignments can be incrementally
evaluated as the decoder runs, or can be stored and applied when the
decoder terminates.

%% \begin{remark}[Half-baked performance idea]
%%   The matcher implementation requires expression evaluators
%%   $\mathit{evalExp}$ and $\mathit{evalBexp}$ which access the $\theta$
%%   being built up for values of already-processed fields. Since
%%   $\theta$ is defined in terms of L-values, it should be
%%   straightforward to model it in the runtime data representation of
%%   whatever compiler the matcher is built on. (Note, however, that
%%   variable-sized arrays do not provide constant-time array indexing.)
%%   If that assumption is true, then expression evaluation---rather than
%%   running in interpreted style over the user-defined $\mathit{exp}$
%%   datatype---might be able to use the host language evaluator on
%%   compiled host language arithmetic and boolean expressions. The
%%   formal details need to be worked out; especially interesting would
%%   be building a correctness story for such a transformation in the
%%   context of the CakeML compiler.
%% \end{remark}

\section{Filtering}

Given the matching algorithm, the building blocks for expressing
message filters are already in place. A simple approach we are
investigating is to add a new ``assertion'' constructor to the
definition of $\tau$:

\[
  \konst{Assert}(\mathit{bexp})
\]

\noindent The semantics of \konst{Assert} can already be expressed
with the notion of an empty record, and $\emptyset$, the empty
language.
\[\konst{Assert}\; P = \konst{Union} \; (P, \{\,\}) \ (\neg P, \emptyset) \]

\noindent The implementation mirrors this closely
\[
\LangTheta{\konst{Assert}\; \mathit{bexp}} =
 \left\{
    \begin{array}{ll}
    \varepsilon, &  \mathrm{if}\ \konst{evalBexp}\;\theta\;\mathit{bexp} = \konst{true} \\
    \emptyset & \mathrm{otherwise}
 \end{array}
 \right.
\]

The $\mathit{bexp}$ type provides a minimal (but adequate for
illustrative purposes) assertion language over R-values. For example,
the following \konst{contig}

\begin{verbatim}
  { A : i32
    B : i32
    C : Assert (A + B <= 42)
  }
\end{verbatim}

\noindent includes field \verb+C+, which is processed after fields
\verb+A+ and \verb+B+, and checks a relationship between the fields.

Mirroring the semantics, almost no change to the matcher function is
needed to support \konst{Assert} forms: the matcher's traversal of
\konst{contig} and string arguments is unchanged, and the accumulation of
contextual information in the form of the map $\theta$ also is
unchanged. Each \konst{Assert} expression is evaluated when it is
encountered; if the result is true, processing continues, otherwise
an error has been found.

Notice that successful match of a message against a \konst{contig} amounts to
a minimal well-formedness check: the message string is long enough to
match the \konst{contig}. \konst{Assert} fields provide a basis for deeper
semantic checks. It is also worth noting that certain features of
\konst{contig}s require checks no matter what kind of user-specified
properties are required: both enumerations and arrays have
side-conditions that \konst{Assert} can enforce.

\begin{example}
\konst{contig}s support enumerations. An \emph{enumeration} declaration
introduces a new base contiguity type, and also adds the specified
elements to the $\Delta$ map associating constant names to numbers.
Suppose that enumerations are allowed to have up to 256 elements,
allowing any enumerated element to fit in one byte. The following
enumeration is taken from the uxAS project \cite{}:
\begin{verbatim}
  NavigationMode
    = Waypoint | Loiter | FlightDirector
    | TargetTrack | FollowLeader | LostComm
\end{verbatim}

A field expecting a \verb+NavigationMode+ element will be 1 byte wide,
and thus there are 250 byte patterns that should not be allowed in the
field. Thus, the \konst{contig}

\begin{verbatim}
  { A : NavigationMode }
\end{verbatim}

should be replaced by

\begin{verbatim}
  { A : NavigationMode
    A-range : Assert (A <= 5)
  }
\end{verbatim}

\noindent in order to specify an on-the-fly check. A pass automatically
inserting such checks into a \konst{contig} would be straightforward to
implement.

\end{example}

\noindent \konst{Assert} expressions can also help with specifications
on array sizes.

\begin{example}

In uxAS messages the length of every array element is held in a
separate \emph{length} field 2 bytes in size. Thus the following
\konst{contig}, in the absence of any further constraint, supports arrays of
length up to 65536 elements, which the receiver system may not have
adequate resources to accept.

\begin{verbatim}
  { len : u16
    elts : i32 [len] }
\end{verbatim}

\noindent In the meta-data for such messages, there can be information
about the maximum allowed size, usually a fairly small number. This
can be directly expressed inside the \konst{contig} with an \konst{Assert}:

\begin{verbatim}
  { len : u16
    len-range : Assert (len <= 8)
    elts : i32 [len] }
\end{verbatim}

\noindent Note that the expected array length should be specified
before the array itself, otherwise the allocation attempt might be
made before the check.
\end{example}


\noindent The following work remains to be done to implement filters
properly.

\begin{itemize}

\item The assertion language must be expressive in order to be useful,
  and must be extensible with user-defined functions and
  well-formedness assertions. In the context of the CASE project, the
  AADL/AGREE/SPLAT toolchain does provide the needed expressiveness,
  but there is work to do in order to port the $\tau$ type and the
  associated algorithms to that context. In particular, $\mathit{exp}$
  only features one kind of numbers, but well-formedness
  specifications will require a wide range of signed and unsigned
  integers, plus floating point.

\item Assertions on arrays require some thought as how best to specify
  universal and existential predicates on arrays.

\item There is a choice as to whether processing should fail as soon
  as one assertion fails vs. going all the way through and keeping
  track of all the failures for a final report. This is very similar
  to the issues surrounding parser error messages in compilers.

\end{itemize}

\section{Extensions and future work}

Various extensions have been easy to add to the contiguity type
framework:

\begin{description}

\item [Enumerations] These have been already mentioned. An enumeration
  is a base type, thus having a fixed width (for us, currently, one
  byte). Each element of the enumeration has a distinct numerical
  value, so the elements are added to the map from constant names to values.

\item [Raw blocks] A raw chunk of a string (byte array) of a size that
  can depend on the values of earlier fields is easy to specify:
  \[ \konst{Raw}\; \mathit{exp} \]
For example, a large \konst{Array} form can lead to a large number of
L-values being stored in $\theta$; if none are ever accessed later, it
can be preferable to simply declare a \konst{Raw} block. Thus a 2D
array could be blocked out in the following manner:

\begin{verbatim}
  { rows : i32
    cols : i32
    block : Raw (rows * cols)
  }
\end{verbatim}

\item [Guest scanners] \konst{Array} provides a sort of \emph{bounded
  iteration} capability, but occasionally a message has fields
  terminated by a sentinel value, and this requires a form of
  \emph{unbounded iteration}. For example, C strings use the first
  ASCII character (0) as the terminal character. Since there are many
  other such conventions that may be used, it seemed useful to provide
  a general ability to host scanning functions. This is accomplished via the following constructor:

  \[ \konst{Scanner}\;
     (\mathit{scanfn} : \konst{string} \to (\konst{string} \times \konst{string})\konst{option}) \]

  When a custom scanner is encountered during the matching process,
  the scanner is invoked on the input and should either fail or
  provide an $(s_1,s_2)$ pair representing a splitting of the
  input. Then $s_1$ is added to $\theta$ at the current \lval, and
  matching continues on $s_2$.

\item [Non-copying implementations] In the discussion so far, we have
  assumed that the input string is being broken up into substrings
  that are placed into the \lval{} map $\theta$. However, very little
  is changed if, instead of a substring, an \lval{} in $\theta$ maps
  to a pair of indices $(\mathit{pos},\mathit{width})$ designating the
  location of the substring. The result is a matcher that never copies
  byte buffer data. This seems to promise being able to synthesize
  efficient filters.

  In making this representation change, there is a slight change to
  the semantics. In the original, $\theta(\lval)$ yields a string
  whereas in the non-copying version, $\theta(\lval)$ yields a pair of
  indices, which means that the original string $\mathit{str}_0$ needs
  to be included in applying the assignment.
\end{description}

\subsection{Current and future work}

\begin{description}

\item [Bits and bytes] Bitcodec \cite{} and other tools support packed
  messages that need to be indexed at the bit-level. Indeed some tools
  support mixtures of bits and bytes. Having a purely bit-centric
  version of contiguity types seems straightforward: since everything
  runs on a notion of \emph{width}, the only change required would be
  to the function that breaks a chunk off the input string.

\item [Expression analysis] There are many opportunities to
  automatically analyze guards of union types for satisfiability and
  disjointness. Also array indexes could be partial, in that an array
  index inside an R-value might not be meaningful. For example, in

\begin{verbatim}
  { len : u16
    A : u8 [len]
    G : char [A[3]]
  }
\end{verbatim}

the reference to \verb+A[3]+ in field \verb+G+ will be incorrect if
\verb+len+ is less than 4. A well-placed \konst{Assert} would be
useful, but static checks would be more helpful in avoiding the
creation of such bad formats.

\item [Left-centricity] Currently, matching is left-centric: only
  fields to the left of the current input position are put in $\theta$
  and can determine what happens next. However, certain formats (\eg,
  IPv4) refer ahead. Since matching visits every field in a \konst{contig}, it
  could be that fields can be topologically sorted on dependency so
  that such anomalies can be handled. Such sorting should be able to
  be justified semantically.

\item [Integration into \konst{SPLAT} machinery] In order to generate
  matchers and filters for various message types needed in CASE, we
  need to formalize contiguity types and the matcher in HOL4, use the
  CakeML translator to generate CakeML ASTs, then invoke the CakeML
  compiler with suitable FFI calls needed to accomplish IO. Ultimately
  what will be needed are implementations of basic maps from bytes
  into the base types of CakeML. Simultaneously, we will have to
  integrate \konst{contig}s into the \konst{SPLAT} AST level, so that AADL and
  AGREE specifications are mapped into the correct \konst{contig}s.

\end{description}

\section{Conclusion}

We have designed and implemented a specification language for message
formats, based on formal languages and the venerable notion of L- and
R-values from imperative programming. The notion of contiguity type
seems to give a lot of expressive power, sufficient to tackle
difficult idioms in self-describing formats. Contiguity types
integrate common structuring mechanisms from programming languages,
such as arrays and records, while keeping the foundation in sets of
strings, which seems appropriate for message specifications. We have
also been able to characterize message parsing as an instance of
pattern matching, which provides clarity in formalizing algorithms and
stating correctness.

\appendix

\section{Examples}

Following are some longer examples of contiguity types.

\begin{enumerate}

\item The following is an abstracted 802-11 MAC header format. The
  format depends on an enumerated type

\begin{verbatim}
 Enum ("Frame",
    [("Management",0),
     ("Control",   1),
     ("Data",      2),
     ("Reserved",  3)
     ])
\end{verbatim}
\noindent The actual MAC header is in terms of bits while our current
implementation is for bytes. (Requires very little to change, as
argued in the body of the paper.) We use the $\konst{Raw}$ construct.

\begin{verbatim}
   {protocol  : Raw(2)
    type      : Frame
    subType   : Raw(4)
    toDS      : Raw(1)
    fromDS    : Raw(1)
    moreFrag  : Raw(1)
    retry     : Raw(1)
    powerMgmt : Raw(1)
    moreData  : Raw(1)
    wep       : Raw(1)
    order     : Raw(1)
    duration  : Raw(16)
    tails     : Union {
     (type = Frame'Data)
          --> {address1   : Raw(48)
               address2   : Raw(48)
               address3   : Raw(48)
               fragNumber : Raw(4)
               seqNumber  : Raw(12)
               address4   : Raw(48)
              }
     (type = Frame'Control and subType = 11)
          --> {receiver    : Raw(48)
               transmitter : Raw(48)
              }
     (type = Frame'Control and subType = 12)
          --> {receiver : Raw(48)
              }
     }
    }
\end{verbatim}

\item uxAS AirVehicleState. Notable in this format is the
  \verb+PayloadStateList+ field which is a variable-length array of
  records. Each record has its own variable-length array of key-value
  pairs, and the key and value of each such pair is a variable-length
  string.

\begin{verbatim}
AirVehicleState =
   {EntityState   : EntityState
    Airspeed      : Float
    VerticalSpeed : Float
    WindSpeed     : Float
    WindDirection : Float
   }
\end{verbatim}
\noindent where
\begin{verbatim}
EntityState =
   {ID : i64
    u : Float
    v : Float
    w : Float
    udot : Float
    vdot : Float
    wdot : Float
    Heading : Float
    Pitch : Float
    Roll : Float
    p : Float
    q : Float
    r : Float
    Course : Float
    Groundspeed : Float
    Location :
      {Latitude : Double
       Longitude : Double
       Altitude : Float
       AltitudeType : AltitudeType
      }
    EnergyAvailable : Float
    ActualEnergyRate : Float
    PayloadStateListLen : u16
    PayloadStateList :
     {PayloadID : i64
      ParametersLen : u16
      Parameters : KeyValuePair [ParametersLen]
      } [PayloadStateListLen]
    CurrentWaypoint : i64
    CurrentCommand : i64
    Mode : NavigationMode
    AssociatedTasksLen : u16
    AssociatedTasks : i64 [AssociatedTasksLen]
    Time : i64
    InfoLen : u16
    Info : KeyValuePair [InfoLen]
    }
\end{verbatim}

\noindent and
\begin{verbatim}
String = {len  : u16
          elts : Char [len]
         }

KeyValuePair =
   {key   : String
    value : String
   }

Enum ("AltitudeType",
  [("AGL",0),
   ("MSL",1)
  ])

Enum ("NavigationMode",
  [("Waypoint",      0),
   ("Loiter",        1),
   ("FlightDirector",2),
   ("TargetTrack",   3),
   ("FollowLeader",  4),
   ("LostComm",      5)
  ])
\end{verbatim}

\end{enumerate}



%% Bibliography
\bibliographystyle{plain}
\bibliography{contig}

\end{document}
